\documentclass[a4paper]{article}
\usepackage[utf8]{inputenc}
\usepackage{amsmath}
\usepackage{amsfonts}
\usepackage{amssymb}
\usepackage{graphicx}
\usepackage{geometry}
\usepackage[english]{babel}
\usepackage{enumitem}%can be used for automatic numbering of requirements/tests.
\usepackage{hyperref}
\usepackage[colorinlistoftodos]{todonotes}
\usepackage{color} % Used to write in red,blue,green etc.
\usepackage{float}
\usepackage[titletoc]{appendix}

\newlength{\testlabellength}
\settowidth{\testlabellength}{Instruction 100.10.10}
\newenvironment{testlist}{\begin{enumerate}[label=\bfseries Instruction \thesubsection.\arabic* , labelindent=0pt, labelwidth=\testlabellength , leftmargin=2cm]}{\end{enumerate}}

\newenvironment{precondition}{
{\color{white}BLARG}\\ 
\textbf{Precondition}
\begin{itemize}[labelindent=0cm, labelwidth=2cm , leftmargin=1cm]
}
{\end{itemize}}

\newenvironment{instruction}{
\textbf{Instructions:}
\begin{enumerate}[label=\bfseries  \arabic*., labelindent=0cm, labelwidth=2cm , leftmargin=1cm]
}
{\end{enumerate}}

\newenvironment{postcondition}{
\textbf{Postcondition:}
\begin{itemize}[labelindent=0cm, labelwidth=2cm , leftmargin=1cm]
}
{\end{itemize}}

\title{SVVI - Software Verification and Validation Instruction}
\newcommand{\version}{v0.0}
\newcommand{\SVVI}{PUSS154215}
\author{Testgroup, Team 2}


%-------------------------TITLE-----------------------------------------

\begin{document}
\begin{titlepage}
\newgeometry{left=2cm,top=1cm,right=2cm}
\newcommand{\HRule}{\rule{\linewidth}{0.5mm}}

\begin{minipage}{0.5\textwidth}
\begin{flushleft} % Responsible persons, write on separate lines
\textit{Responsible for this document:}\\
Oskar Fällström %Not entirely sure who this should be, possibly project leaders as well?
\end{flushleft}
\end{minipage}
~
\begin{minipage}{0.4\textwidth}
\begin{flushright}
\SVVI\ \version\ %Dokumentnummer enl. projekthandledning s. 22-23 och insidan av pärmen
\today
\end{flushright}
\end{minipage}\\[3cm]

\centering
\textsc{\LARGE Team 2}\\[0.5cm]

\HRule \\[0.4cm]
{ \huge \bfseries SVVI - Software Verification and Validation Instruction }\\[0.4cm] % Title of your document
\HRule \\[1.5cm]

\vfill
\begin{flushleft}
\textit{Authors of this document:}\\
Måns Andersson \\
Hanna Autio \\
Moa Eklöf \\
Oskar Fällström \\
Ulf Hörndahl \\
Jonathan Lundholm
\end{flushleft}


\end{titlepage}
\pagenumbering{gobble}

\begin{center}
\textit{\large Version History}

    \begin{tabular}{ | l | l | l | p{5cm} |}
    \hline
    \textbf{Version}		& \textbf{Date}		& \textbf{Responsible}					& \textbf{Description}					\\ \hline
    0.0						& 150916 			& UH									& Document created						\\ \hline
    \end{tabular}
\end{center}

\setcounter{tocdepth}{2}
\tableofcontents
\newpage
\pagenumbering{arabic}


%--------------------Reference Documents ---------------------------------------
\section{Reference Documents}
\begin{enumerate}
\item PUSS154212 - System Requirements Specification for the current project \label{refdocs:srs}
\item Programvaruutveckling för Stora System - Projekthandledning v2.2 (\textit{Institutionen för datavetenskap}, Lunds Univeritet 2015) \label{refdocs:projekthandledning}
\item PUSS154213 - Software Verification and Validation Specification \label{redocs:SVVS}
\item PUSS154253 - Test Matrices for SVVS. \label{refdocs:matrices}
\end{enumerate}

%--------------------Introduction ---------------------------------------
\section{Introduction}
This document contains the test instructions for the tests specified in PUSS154213 (ref. \ref{redocs:SVVS}). The test instructions are in appendices \ref{appendix:section:functiontest} and \ref{appendix:section:systemtest}. The numbering corresponds to that in the SVVS (ref. \ref{redocs:SVVS})


%--------------------Appendix A: Function test Specification ----------------
\newpage
\begin{appendices}

\section{Function Test Instructions} \label{appendix:section:functiontest}
This is appendix \ref{appendix:section:functiontest} where we list all function test instructions.

\subsection{MyDevices View Test Instructions}
\begin{testlist}

    \item 
    	\begin{precondition}
    		\item Telephone is on
    		\item Application installed but not running.
    	\end{precondition}
    	\begin{instruction}
    		\item Locate the application icon on the telephone.
    		\item With your right index finger press the application icon.
	    	\item Wait.
    		\item Look at the screen and confirm the resemblance with figure \textbf{blurp} in \ref{refdocs:srs}.
    	\end{instruction}
    	\begin{postcondition}
    		\item The application is running and the MyDevices screen is shown.
    	\end{postcondition}
    %Test description: The first screen that is shown on startup is the MyDevicesView (Req. 5.2.1).
    
	\item
		\begin{precondition}
			\item Telephone is on
			\item Application is not running.
    	\end{precondition}
    	\begin{instruction}
    		\item Start the application
    	\end{instruction}
    	\begin{postcondition}
    		\item The list of available devices is empty.
    	\end{postcondition}
    %The list of available devices is empty on application start up. (Req. 5.2.2)
    
    \item	
    	\begin{precondition}
    		\item The application is running 
    		\item My Devices View is shown
    		\item At least one device is displayed in the list
    	\end{precondition}
    	\begin{instruction}
			\item Drag the list downward
		\end{instruction}
		\begin{postcondition}
			\item The list is not moving but a "THING" is showing that there is no more devices further up in the list (instruktion?).
		\end{postcondition}
   %\item The list on the MyDevices view is scrollable. (Req. 5.2.3)
    
    \item
    	\begin{precondition}
    		\item The application is running
    		\item The MyDevices View is shown
    		\item At least one device is displayed in the list (instruktion?).
    	\end{precondition}
		\begin{instruction}
			\item Touch one device
		\end{instruction}
		\begin{postcondition}
			\item Background color of the chosen device changes (and the "Control Device"-button is enabled.?)
		\end{postcondition}
    %\item The devices on the list are selectable. (Req 5.2.4)
    
    \item 
   		\begin{precondition}
   			\item The application is running
   			\item The MyDevices View is shown
   			\item At least two devices are displayed in the list (instruktion?)
   			\item One device is selected (instruktion?).
   		\end{precondition}
   		\begin{instruction}
   			\item Touch another device
   		\end{instruction}
   		\begin{postcondition}
   			\item The background color of the last chosen device is different to all other devices in the list.
   		\end{postcondition}
                % Räcker det med att testa bakgrundfärg eller måste man även prova att trycka på "Control device"?? Hanna: Om färgen ska vara det som indikerar vilken device som är selected borde det räcka.
    %\item Only one device can be selected at a time. (Req. 5.2.5)
    
    \item
    	\begin{precondition}
    		\item The application is running
    		\item The MyDevices View is shown
    		\item No device is selected
    	\end{precondition}
    	\begin{instruction}
    		\item Touch the "Control Device"-button.
    	\end{instruction}
    	\begin{postcondition}
    		\item A pop-up message "Please select a device" is displayed.
    	\end{postcondition}
    %\item When no device is selected and the "Control Device" -button is pressed, a pop-up message "Please select a device" is displayed. (Req. 5.2.6)
   
	\item
		\begin{precondition}
			\item
		\end{precondition}
		\begin{instruction}
			\item
		\end{instruction}
		\begin{postcondition}
			\item
		\end{postcondition}
		   
    %\item Sensors are displayed in the list of available devices with "Sensor" as name, and it's MAC address as address and their indentifier as id. (Req. 5.2.7)
   
	\item
		\begin{precondition}
			\item
		\end{precondition}
		\begin{instruction}
			\item
		\end{instruction}
		\begin{postcondition}
			\item
		\end{postcondition}
		   
    %\item Light bulbs are displayed in the list of available devices with "Light Bulb" as name and it's MAC address as address and their indentifier as id. (Req. 5.2.8)
   
	\item
		\begin{precondition}
			\item
		\end{precondition}
		\begin{instruction}
			\item
		\end{instruction}
		\begin{postcondition}
			\item
		\end{postcondition}
		   
    %\item The "Get Devices" -button performs a scan for available devices when pressed. (Req. 5.2.9)
   
	\item
		\begin{precondition}
			\item
		\end{precondition}
		\begin{instruction}
			\item
		\end{instruction}
		\begin{postcondition}
			\item
		\end{postcondition}
		   
   %\item When the back button is pressed the application is closed (Req. 5.2.10)
    
    % Nytt requirement!
    \item
		\begin{precondition}
			\item
		\end{precondition}
		\begin{instruction}
			\item
		\end{instruction}
		\begin{postcondition}
			\item
		\end{postcondition}
		   
    %\item The layout of the MyDevices View resemble figure 1 in appendix  Ref \ref{refdocs:srs} (5.2.11).
\end{testlist}

%--------------------------Sensor view tests --------------------------------------
\subsection{Sensor View Test Instructions}
\begin{testlist}
	\item
 		\begin{precondition}
 			\item The MyDevices View is opened
 			\item A sensor device populates the list
 		\end{precondition}
 		\begin{instruction}
 			\item Select a sensor device
 			\item Push the Control device button
 		\end{instruction}
 		\begin{postcondition}
 			\item The Sensor View is opened. 
 		\end{postcondition}

%When the "Control Device"-button in the MyDevices View is pressed and a sensor is selected, the Sensor view is opened (Req. 5.3.1).

	\item
		\begin{precondition}
			\item The Sensor View is opened.
		\end{precondition}
		\begin{postcondition}
			\item The sensor name and mac-adress is shown in the top of the view.
		\end{postcondition}
%\item The sensor name and mac-address is shown in the top of the view (Req. 5.3.2, 5.3.3).

	\item
		\begin{precondition}
			\item The Sensor View is opened and
			\item The on/off-status is set to off.
		\end{precondition}
		\begin{instruction}
			\item Switch the on/off-status to on.  
	  	    \item Control that the status lamp on the device is "shining"
	  	\end{instruction}
	  	\begin{postcondition}
	  		\item The on/off -status of the selected sensor have been changed to on.
	  	\end{postcondition}
%\item It is possible to change the on/off -status of the selected sensor with a switch (Req. 5.3.4).

	\item
		\begin{precondition}
			\item The Sensor View is opened. 
		\end{precondition}
		\begin{instruction}
			\item Check that there are text field preceded by "T", "P", "H", "M", "G" and "A" in this specified order.
			\item Check that
			\item Wait
		\end{instruction}
		\begin{postcondition}
			\item There are text fields preceded by "T", "P", "H", "M", "G", "A".
		\end{postcondition}
%\item There are text fields preceded by "T", "P", "H", "M", "G", "A" that are used to, respectively, display temperature, pressure, humidity, magnetic field strength, gyroscopic data and acceleration (Req. 5.3.5, 5.3.6, 5.3.7, 5.3.8, 5.3.9, 5.3.10).

	\item
		\begin{precondition}
			\item The Sensor View is opened
			\item The on/off-status is set to on.
		\end{precondition}
		\begin{instruction}
			\item Perform the steps (b)-(c) for the text field preceded by "T", "P", "H", "M", "G" and "A" in this specified order.
			\item Press the corresponding "Get"-button to the text field
			\item Check that the value for the current sensor data are retrieved "if available" and displayed in the corresponding field.
		\end{instruction}
		\begin{postcondition}
			\item The values of the temperature, pressure, humidity, magnetic field strength, qyroscopic and acceleration sensors are retrieved if available and displayed.
		\end{postcondition}

%\item By pressing the corresponding "Get"-button, the values of the temperature, pressure, humidity, magnetic field strength, qyroscopic and acceleration sensors are retrieved if available and displayed (Req. 5.3.11, 5.3.12, 5.3.13, 5.3.14, 5.3.15, 5.3.16).

	\item
		\begin{precondition}
			\item The Sensor View is opened
			\item The on/off-status is set to on.
		\end{precondition}
		\begin{instruction}
			\item Push the "Get All"-button.
		\end{instruction}
		\begin{postcondition}
			\item The value for all six sensors data are displayed in their corresponding "text" fields.
		\end{postcondition}
%\item The "Get All" -button gets the values for all six sensors and displays them (Req. 5.3.17).

	\item
		\begin{precondition}
			\item The Sensor View is opened
			\item Sensors data are displayed in the text fields.
		\end{precondition}
		\begin{instruction}
			\item Push the "Clear All"-button.
		\end{instruction}
		\begin{postcondition}
			\item All the sensor text fields are cleared from data/value.
		\end{postcondition}
%\item The "Clear All" button clears all sensor text fields from data (Req. 5.3.18).

	\item
		\begin{precondition}
			\item The Sensor View is opened
			\item The on/off-status is set to on
		\end{precondition}
		\begin{instruction}
			\item Perform the steps (b)-(c) for the text field preceded by "T", "P", "H", "M", "G" and "A" in this specified order.
    		\item Check that 
	    	\item Wait.
    	\end{instruction}
    	\begin{postcondition}
    		\item There are text fields preceded by "T", "P", "H", "M", "G", "A".
    	\end{postcondition}
%\item If there is no data to retrieve for any of the physical quantities: temperature, pressure, humidity, magnetic field strength, gyroscopic data and acceleration when the corresponding "Get" button is pressed the corresponding, text fields display "No data available" (Req. 5.3.19, 5.3.20, 5.3.21, 5.3.22, 5.3.23, 5.3.24).

%	\item
%		\begin{precondition}
%			\item
%		\end{precondition}
%		\begin{instruction}
%			\item
%		\end{instruction}
%		\begin{postcondition}
%			\item
%		\end{postcondition}
		   
	\item The on/off-switch is set according to the information from the REST API (Req. 5.3.25).

%	\item
%		\begin{precondition}
%			\item
%		\end{precondition}
%		\begin{instruction}
%			\item
%		\end{instruction}
%		\begin{postcondition}
%			\item
%		\end{postcondition}
		   
	\item When the back button is pressed in the Sensor View, the MyDevices View is opened (Req. 5.3.26).

%	\item
%		\begin{precondition}
%			\item
%		\end{precondition}
%		\begin{instruction}
%			\item
%		\end{instruction}
%		\begin{postcondition}
%			\item
%		\end{postcondition}
		   
	\item When the sensor view is opened the temperature, pressure, humidity, magnetic field strength, gyroscopic data and acceleration text fields are empty (Req. 5.3.27).

%	\item
%		\begin{precondition}
%			\item
%		\end{precondition}
%		\begin{instruction}
%			\item
%		\end{instruction}
%		\begin{postcondition}
%			\item
%		\end{postcondition}
		   
	\item The layout of the Sensor View resemble figure 2 in appendix  Ref \ref{refdocs:srs} (Req. 5.3.28).

\end{testlist}

\subsection{Light Bulb View Test Instructions}
\begin{testlist}

    \item
    	\begin{precondition}
    		\item The application is in the MyDevices View
    		\item At least one lightbulb is in the list of available devices.
    	\end{precondition}
    	\begin{instruction}
    			\item Choose a light bulb in the list of available devices.
    			\item Press the "Control Device" button
    	\end{instruction}
    	\begin{postcondition}
    		\item The Light Bulb View is open
    	\end{postcondition}

	%\item The Light Bulb View opens when a light bulb is chosen in the MyDevices View and the "Control device" button is pressed. (Req 5.4.1)
	
	\item
		\begin{precondition}
			\item The Light Bulb View is shown
		\end{precondition}
    	\begin{postcondition}
    		\item The name and mac-address of the light bulb is shown at the top of the view.
    	\end{postcondition}
    	
	%\item The name of the selected light bulb is shown at the top of the View (Req 5.4.2).
	
	\item
		\begin{precondition}
			\item The Light Bulb View is open
			\item The switch to control the on/off-status of the light bulb is set to on (off)
			\item The state of the light bulb corresponds to the state of the switch
		\end{precondition}
    	\begin{instruction}
    		\item Change the switch to off (on)
    	\end{instruction}
    	\begin{postcondition}
    		\item The switch to control the light bulb is set tp off (on)
    		\item The light bulb turns off (on).
    	\end{postcondition}
    		
  
    % \item The selected light bulb can be turned on/off with a switch (Req. 5.4.3). 
    
   	\item
   		\begin{precondition}
   			\item The Light Bulb View is open
		\end{precondition}
    	\begin{instruction}
    		\item For each of the four fields, enter at least one character
    	\end{instruction}
    	\begin{postcondition}
    		\item The four fields are preceded by "R:", "G:", "B:" and "W:" respectively
    		\item Is is possible to enter a character into the fields
    	\end{postcondition}

    % \item The field of R-, G-, B-, W-value is editable and preceded by "R:", "G:", "B:" and "W:" respectively (Req. 5.4.4, 5.4.5, 5.4.6, 5.4.7). 
	
	\item
		\begin{precondition}
			\item The MyDevices View is open
			\item There is a light bulb in the list of available devices
		\end{precondition}
		\begin{instruction}
			\item Select a light bulb
			\item Press the "Control Device" button
		\end{instruction}
		\begin{postcondition}
			\item The Light Bulb View is opened
			\item The fields specified in Req. 5.4.4-5.4.7 ref. \ref{refdocs:srs} are empty
		\end{postcondition}	   
%	\item When the Light Bulb View is opened, the fields are empty (Req. 5.4.8).

	\item
		\begin{precondition}
			\item The Light Bulb View is open
			\item The light bulb is on
		\end{precondition}
		\begin{instruction}
			\item Press the "Get"-button
		\end{instruction}
		\begin{postcondition}
			\item The R-, G-, B-, W-values are displayed in the fields specified in Req. 5.4.4-5.4.7 in ref \ref{refdocs:srs}
		\end{postcondition}
%	\item The "Get"-button retrieves the R-, G-, B-, W-values and present them in their corresponding fields (Req. 5.4.9). 

	\item This test should be run 5 times with the following configurations:
		\begin{itemize}
			\item A=FF, B=00, C=00, D=00 (color: Red)
			\item A=00, B=FF, C=00, D=00 (color: Green)
			\item A=00, B=00, C=FF, D=00 (color: Blue)
			\item A=00, B=00, C=00, D=FF (color: White)
			\item A=10, B=10, C=10, D=10 (color: White)
		\end{itemize}
		\begin{precondition}
			\item The Light Bulb View is open
			\item The light bulb is on
		\end{precondition}
		\begin{instruction}
			\item Set the "R:"-field to A
			\item Set the "G:"-field to B
			\item Set the "B:"-field to C
			\item Set the "W:"-field to D
			\item Press the "Set"-button
		\end{instruction}
		\begin{postcondition}
			\item The light bulb has the specified color
		\end{postcondition}
		
%	\item The "Set button" sets the color of the light bulb (Req. 5.4.10).

	\item This test should be run 4 times with the following configurations:
		\begin{itemize}
			\item A blank, B=FF, C=00, D=00 (color: Green)
			\item A=00, B blank, C=FF, D=00 (color: Blue)
			\item A=00, B=00, C blank, D=FF (color: White)
			\item A=FF, B=00, C=00, D blank (color: Red)
		\end{itemize}
		\begin{precondition}
			\item The Light Bulb View is open
			\item The light bulb is on
			\item The R-, G-, B-, W-, fields show FF, FF, FF and FF respectively
			\item The light bulb glows white
		\end{precondition}
		\begin{instruction}
			\item Set the "R:"-field to A
			\item Set the "G:"-field to B
			\item Set the "B:"-field to C
			\item Set the "W:"-field to D
			\item Press the "Set"-button
			\item Press the "Get"-button
		\end{instruction}
		\begin{postcondition}
			\item The light bulb has the specified color
			\item The fields that were left blank show "00"
		\end{postcondition}
		
%	\item If an input value is left blank the value is interpreted as 00 (Req. 5.4.11) .   

	\item This test should be run twice with the following configurations:
		\begin{itemize}
			\item A = 'AAA'
			\item A = 'EE'
		\end{itemize}
		\begin{precondition}
			\item The Light Bulb View is open
		\end{precondition}
		\begin{instruction}
			\item Enter A into the "R:"-field
			\item Enter A into the "G:"-field
			\item Enter A into the "B:"-field
			\item Enter A into the "W:"-field
		\end{instruction}
		\begin{postcondition}
			\item The value A is not entered
		\end{postcondition}
%    \item The fields only accepts two characters that represent hexadecimal numbers (e.g. 00 to FF and all combinations inbetween) (Req 5.4.12, Req 5.4.13).

	\item
		\begin{precondition}
			\item The Light Bulb View is open
			\item The light bulb is on
			\item The light bulb glows white
		\end{precondition}
		\begin{instruction}
			\item Set the "R:"-field to FF
			\item Set the "G:"-field to 00
			\item Set the "B:"-field to 00
			\item Set the "W:"-field to 00
			\item Press the "Set"-button
		\end{instruction}
		\begin{postcondition}
			\item A pop-up message saying "Color successfully changed" is displayed
			\item The light bulb is red
		\end{postcondition}
%    \item A pop-up message saying "Color successfully changed" is displayed when the values of the light bulb were successfully set. (Req 5.4.14)

%	\item
%		\begin{precondition}
%			\item
%		\end{precondition}
%		\begin{instruction}
%			\item
%		\end{instruction}
%		\begin{postcondition}
%			\item
%		\end{postcondition}
    \item A pop-up message saying "Error: Could not change color." is displayed when the values of the light bulb were unsuccessfully set. (Req 5.4.15)

	\item
		\begin{precondition}
			\item The Light Bulb View is open
			\item The switch to control the on/off-status of the light bulb is set to off
			\item The light bulb is turned off.
		\end{precondition}
		\begin{postcondition}
			\item The "Set"-button is unavailable
		\end{postcondition}
%    \item When the light bulb is off, the "Set"-button is unavailable (Req. 5.4.16).

	\item
		\begin{precondition}
			\item The Light Bulb View is open
		\end{precondition}
		\begin{instruction}
			\item Press the back button
		\end{instruction}
		\begin{postcondition}
			\item The MyDevices View is open
		\end{postcondition}
%    \item When the back button is pressed the system switches to MyDevices View (Req. 5.4.17).
    
	\item
		\begin{precondition}
			\item The Light Bulb View is open
		\end{precondition}
		\begin{postcondition}
			\item The layout of the screen resembles figure 3 in appendix A in Ref \ref{refdocs:srs}
		\end{postcondition}
%    \item The layout of the Light Bulb View resemble figure 3 in appendix  Ref \ref{refdocs:srs} (Req. 5.4.18).

\end{testlist}	

%--------------------Appendix B: System Test Specification ----------------
\newpage

\section{System Test Instruction} \label{appendix:section:systemtest}
This is appendix \ref{appendix:section:systemtest} where we list all system test specifications.

\subsection{Use Cases}
\begin{testlist}
	\item 
		\begin{precondition}
			\item The application is running and MyDevices view is displayed.
			\item There is a lightbulb and a sensor device within scan range of the MVD.
			\item No other devices are within range of the MVD.
		\end{precondition}
		\begin{instruction}
			\item With your right index finger press the \emph{Get Devices} button.
			\item Verify that the Light bulb is shown.
			\item Verify that the Sensor device is shown.
		\end{instruction}
		\begin{postcondition}
			\item The light bulb is displayed in the MyDevices view.
		 	\item The sensor device is displayed in the MyDevices view.
		\end{postcondition}
	%SVVS: Scenario 5.1.1 in Ref \ref{refdocs:srs} is supported (Req. 5.1.1).

	\item 
		\begin{precondition}
			\item The application is running and MyDevices view is displayed.
			\item No devices are within range of the MVD.
		\end{precondition}
		\begin{instruction}
			\item With your right index finger press the \emph{Get Devices} button.
			\item Verify that a pop-up message with the text \emph{No devices found} is shown.
		\end{instruction}
		\begin{postcondition}
			\item No devices are connected.
		\end{postcondition}
	%SVVS: Exception 1 described in scenario 5.1.1 can be generated by removing the devices from range while attempting scenario 5.1.1 (Req 5.1.1).
	
%	\item
%		\begin{precondition}
%			\item
%		\end{precondition}
%		\begin{instruction}
%			\item
%		\end{instruction}
%		\begin{postcondition}
%			\item
%		\end{postcondition}
	\item Scenario 5.1.2 in Ref \ref{refdocs:srs} is supported (Req. 5.1.2).

%	\item
%		\begin{precondition}
%			\item
%		\end{precondition}
%		\begin{instruction}
%			\item
%		\end{instruction}
%		\begin{postcondition}
%			\item
%		\end{postcondition}	
	\item Exception 1 in scenario 5.1.2 can be generated by not selecting a device in step 1 (Req. 5.1.2).

%	\item
%		\begin{precondition}
%			\item
%		\end{precondition}
%		\begin{instruction}
%			\item
%		\end{instruction}
%		\begin{postcondition}
%			\item
%		\end{postcondition}
	\item Scenario 5.1.3 in Ref \ref{refdocs:srs} is supported (Req. 5.1.3).

%	\item
%		\begin{precondition}
%			\item
%		\end{precondition}
%		\begin{instruction}
%			\item
%		\end{instruction}
%		\begin{postcondition}
%			\item
%		\end{postcondition}
	\item Exception 1 described in scenario 5.1.3 can be generated by not selecting a device in step 1 (Req. 5.1.3). 

%	\item
%		\begin{precondition}
%			\item
%		\end{precondition}
%		\begin{instruction}
%			\item
%		\end{instruction}
%		\begin{postcondition}
%			\item
%		\end{postcondition}
	\item Scenario 5.1.4 in Ref \ref{refdocs:srs} is supported (Req. 5.1.4).

%	\item
%		\begin{precondition}
%			\item
%		\end{precondition}
%		\begin{instruction}
%			\item
%		\end{instruction}
%		\begin{postcondition}
%			\item
%		\end{postcondition}
	\item Scenario 5.1.5 in Ref \ref{refdocs:srs} is supported (Req. 5.1.5).

%	\item
%		\begin{precondition}
%			\item
%		\end{precondition}
%		\begin{instruction}
%			\item
%		\end{instruction}
%		\begin{postcondition}
%			\item
%		\end{postcondition}
	\item Scenario 5.1.6 in Ref \ref{refdocs:srs} is supported (Req. 5.1.6).

%	\item
%		\begin{precondition}
%			\item
%		\end{precondition}
%		\begin{instruction}
%			\item
%		\end{instruction}
%		\begin{postcondition}
%			\item
%		\end{postcondition}
	\item Scenario 5.1.7 in Ref \ref{refdocs:srs} is supported (Req. 5.1.7).

%	\item
%		\begin{precondition}
%			\item
%		\end{precondition}
%		\begin{instruction}
%			\item
%		\end{instruction}
%		\begin{postcondition}
%			\item
%		\end{postcondition}
	\item Scenario 5.1.8 in ref \ref{refdocs:srs} is supported (Req. 5.1.8).

%	\item
%		\begin{precondition}
%			\item
%		\end{precondition}
%		\begin{instruction}
%			\item
%		\end{instruction}
%		\begin{postcondition}
%			\item
%		\end{postcondition}
	\item Scenario 5.1.9 in Ref \ref{refdocs:srs} is supported (Req. 5.1.9).

%	\item
%		\begin{precondition}
%			\item
%		\end{precondition}
%		\begin{instruction}
%			\item
%		\end{instruction}
%		\begin{postcondition}
%			\item
%		\end{postcondition}
	\item If some data is unavailable when attempting scenario 5.1.9, the text in the corresponding field is set to "No data available." (Req. 5.1.9).

%	\item
%		\begin{precondition}
%			\item
%		\end{precondition}
%		\begin{instruction}
%			\item
%		\end{instruction}
%		\begin{postcondition}
%			\item
%		\end{postcondition}
	\item Scenario 5.1.10 in Ref \ref{refdocs:srs} is supported (Req. 5.1.10).

%	\item
%		\begin{precondition}
%			\item
%		\end{precondition}
%		\begin{instruction}
%			\item
%		\end{instruction}
%		\begin{postcondition}
%			\item
%		\end{postcondition}
	\item If some data is unavailable when attempting scenario 5.1.10, the text in the corresponding field is set to "No data available." (Req. 5.1.10).

%	\item
%		\begin{precondition}
%			\item
%		\end{precondition}
%		\begin{instruction}
%			\item
%		\end{instruction}
%		\begin{postcondition}
%			\item
%		\end{postcondition}
	\item Scenario 5.1.11 in Ref \ref{refdocs:srs} is supported (Req. 5.1.11).

%	\item
%		\begin{precondition}
%			\item
%		\end{precondition}
%		\begin{instruction}
%			\item
%		\end{instruction}
%		\begin{postcondition}
%			\item
%		\end{postcondition}
	\item If some data is unavailable when attempting scenario 5.1.11, the text in the corresponding field is set to "No data available." (Req. 5.1.11).

%	\item
%		\begin{precondition}
%			\item
%		\end{precondition}
%		\begin{instruction}
%			\item
%		\end{instruction}
%		\begin{postcondition}
%			\item
%		\end{postcondition}
	\item Scenario 5.1.12 in Ref \ref{refdocs:srs} is supported (Req. 5.1.12).
\end{testlist}

\subsection{Quality Test Instructions}

\begin{testlist}

%	\item
%		\begin{precondition}
%			\item
%		\end{precondition}
%		\begin{instruction}
%			\item
%		\end{instruction}
%		\begin{postcondition}
%			\item
%		\end{postcondition}
	\item Test of usability using persons without prior knowledge of the project or the app (Req. 6.1).

%	\item
%		\begin{precondition}
%			\item
%		\end{precondition}
%		\begin{instruction}
%			\item
%		\end{instruction}
%		\begin{postcondition}
%			\item
%		\end{postcondition}
	\item The response time for any interaction with the system is below two seconds (Req. 6.2).

%	\item
%		\begin{precondition}
%			\item
%		\end{precondition}
%		\begin{instruction}
%			\item
%		\end{instruction}
%		\begin{postcondition}
%			\item
%		\end{postcondition}
	\item The time-out time for the system is 15 seconds (Req. 6.3). 
	
\end{testlist}

\clearpage

\end{appendices}


\end{document}