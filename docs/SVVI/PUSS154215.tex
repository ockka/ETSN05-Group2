\documentclass[a4paper]{article}
\usepackage[utf8]{inputenc}
\usepackage{amsmath}
\usepackage{amsfonts}
\usepackage{amssymb}
\usepackage{graphicx}
\usepackage{geometry}
\usepackage[english]{babel}
\usepackage{enumitem}%can be used for automatic numbering of requirements/tests.
\usepackage{hyperref}
\usepackage[colorinlistoftodos]{todonotes}
\usepackage{color} % Used to write in red,blue,green etc.
\usepackage{float}
\usepackage[titletoc]{appendix}

\newlength{\testlabellength}
\settowidth{\testlabellength}{Instruction 100.10.10}
\newenvironment{testlist}{\begin{enumerate}[label=\bfseries Instruction \thesubsection.\arabic* , labelindent=0pt, labelwidth=\testlabellength , leftmargin=2cm]}{\end{enumerate}}

\newenvironment{config}[1][2]{
{\color{white}BLARG}\\ 
This test should be run #1 times with the following configurations:
\begin{enumerate}[label=\bfseries  \arabic*., labelindent=0cm, labelwidth=2cm , leftmargin=1cm]
}
{\end{enumerate}}

\newenvironment{precondition}{
{\color{white}BLARG}\\ 
\textbf{Precondition}
\begin{itemize}[labelindent=0cm, labelwidth=2cm , leftmargin=1cm]
}
{\end{itemize}}

\newenvironment{instruction}{
\textbf{Instructions:}
\begin{enumerate}[label=\bfseries  \arabic*., labelindent=0cm, labelwidth=2cm , leftmargin=1cm]
}
{\end{enumerate}}

\newenvironment{postcondition}{
\textbf{Postcondition:}
\begin{itemize}[labelindent=0cm, labelwidth=2cm , leftmargin=1cm]
}
{\end{itemize}}

\title{SVVI - Software Verification and Validation Instruction}
\newcommand{\version}{v1.6}
\newcommand{\SVVI}{PUSS154215}
\author{Testgroup, Team 2}


%-------------------------TITLE-----------------------------------------

\begin{document}
\begin{titlepage}
\newgeometry{left=2cm,top=1cm,right=2cm}
\newcommand{\HRule}{\rule{\linewidth}{0.5mm}}

\begin{minipage}{0.5\textwidth}
\begin{flushleft} % Responsible persons, write on separate lines
\textit{Responsible for this document:}\\
Oskar Fällström %Not entirely sure who this should be, possibly project leaders as well?
\end{flushleft}
\end{minipage}
~
\begin{minipage}{0.4\textwidth}
\begin{flushright}
\SVVI\ \version\ %Dokumentnummer enl. projekthandledning s. 22-23 och insidan av pärmen
\today
\end{flushright}
\end{minipage}\\[3cm]

\centering
\textsc{\LARGE Team 2}\\[0.5cm]

\HRule \\[0.4cm]
{ \huge \bfseries Software Verification and Validation Instruction}\\[0.4cm] % Title of your document
\HRule \\[1.5cm]

\vfill
\begin{flushleft}
\textit{Authors of this document:}\\
Måns Andersson \\
Hanna Autio \\
Moa Eklöf \\
Oskar Fällström \\
Ulf Hörndahl \\
Jonathan Lundholm
\end{flushleft}


\end{titlepage}
\pagenumbering{gobble}

\begin{center}
\textit{\large Version History}

    \begin{tabular}{ | l | l | l | p{5cm} |}
    \hline
    \textbf{Version}		& \textbf{Date}		& \textbf{Responsible}					& \textbf{Description}					\\ \hline
    1.0						& 2015-10-06			& HA									& Baseline								\\ \hline
    1.1						& 2015-10-08			& HA									& Changes according to PR8				\\ \hline
    1.2						& 2015-10-08			& HA									& Changes according to PR12 		\\ \hline
    1.3						& 2015-10-08			& HA									& Changes according to PR13		\\ \hline
    1.4						& 2015-10-08			& OF									& Changes according to SRS v.1.4		\\ \hline
    1.5						& 2015-10-09			& OF									& Changes according to PR19	\\ \hline
    1.6						& 2015-10-14			& UH									& Changed test instruction in B.1.5. \\ \hline
    \end{tabular}
\end{center}

\setcounter{tocdepth}{2}
\tableofcontents
\newpage
\pagenumbering{arabic}


%--------------------Reference Documents ---------------------------------------
\section{Reference Documents}
\begin{enumerate}
\item PUSS154212 - System Requirements Specification v1.4 \label{refdocs:srs} 
\item Programvaruutveckling för Stora System - Projekthandledning v2.2 \newline (\textit{Institutionen för datavetenskap}, Lunds Univeritet 2015) \label{refdocs:projekthandledning}
\item PUSS154213 - Software Verification and Validation Specification v1.4 \label{redocs:SVVS}
\item PUSS154253 - Test Matrices for SVVS v1.2 \label{refdocs:matrices}
\end{enumerate}

%--------------------Introduction ---------------------------------------
\section{Introduction}
This document contains the test instructions for the tests specified in PUSS154213 (ref. \ref{redocs:SVVS}). The test instructions are in appendices \ref{appendix:section:functiontest} and \ref{appendix:section:systemtest}. The numbering corresponds to that in the SVVS (ref. \ref{redocs:SVVS}).
% should the references have "Ref." instead? We have used both in the SVVS...

\subsection{Terminology}
\begin{description}
\item[Device] refers to a Light Bulb or a Sensor Device that the application can interact with.
\item[Device available] is used when a device is connected to the MVD.
\item[Swiping] or \textbf{to drag} is meant to be a relatively smooth motion with a finger (or similar device) on the screen of the telephone.
\item[Off] or \textbf{On} refers to the device's state of transmitting data (in case of the sensor) or light (in case of the light bulb), not whether the device is receiving electrical power or not.
\item[Light bulb] refers to the physical object that shines that we can interact with, while the \textbf{Light} refers to photons emitted from the  \textit{light bulb}.
\item[Controlled environment] refers to a environment where we can control at least one of the following physical observables:
	\begin{itemize}
	\item Temperature
	\item Pressure
	\item Humidty
	\item Magnetic field
	\item Gyroscopic data
	\item Acceleration
	\end{itemize}
\item[Environmental variable] refers to one of the physical observables listed above.
\item[Scenario] refers to the corresponding scenario in \ref{refdocs:srs}.
\end{description}
%--------------------Appendix A: Function test Specification ----------------
\newpage
\begin{appendices}

\section{Function Test Instructions} \label{appendix:section:functiontest}
This is appendix \ref{appendix:section:functiontest} where we list all function test instructions.

%How much should we imply? Is it "touch" a thing or is it "press" a button; should we specify how to select a device from a list by saying "touch the name of the device in the list" or should we just say "select"

\subsection{MyDevices View Test Instructions}
\begin{testlist}

    \item Test of startup screen of application. \label{MyDevicesTest1}
    	\begin{precondition}
    		\item The application is not running
    	\end{precondition}
    	\begin{instruction}
			\item Start the application
    	\end{instruction}
    	\begin{postcondition}
    		\item The application is running
    		\item The MyDevices View is open
    		\item The list of available devices is empty
    	\end{postcondition}
    %Test description: The first screen that is shown on startup is the MyDevicesView (Req. 5.2.1).
    
    
	\item Test of device list on startup. \\
		Instructions for this test have been incorporated into \ref{MyDevicesTest1}
    
    %The list of available devices is empty on application start up. (Req. 5.2.2)

    \item Verification that devices are presented in a list.
    	\begin{precondition}
    		\item MyDevices View is open
    		\item At least two devices are available
            %tankar: Kan listan reagera olika på om det finns en eller flera enheter i den?
            % Funkar det att dra i en lista med bara ett listobjekt?
    	\end{precondition}
    	
    	\begin{instruction}
			\item Verify that the devices are in a list
		\end{instruction}
		\begin{postcondition}
			\item The devices are in a list
            %hur mäter vi att den svarar på input? Om den är instabil och rör sig mer än vad vi petat på den, då är testet godkänt, men appen är ju kass. Likaså om man drar över hela skärmen och den rör sig bara ett par millimeter, så är ju testet passerat, men appen är ju helt omöjlig att använda?
		\end{postcondition}
   %\item The list on the MyDevices view is scrollable. (Req. 5.2.3)
   \newpage
    \item Test of selectability of devices in list.
     	\begin{precondition}
    		\item The MyDevices View is open
    		\item At least one device is available
    	\end{precondition}
		\begin{instruction}
			\item Select one device
		\end{instruction}
		\begin{postcondition}
			\item The device is highlighted and the "Control Device"-button is enabled
		\end{postcondition}
    %\item The devices on the list are selectable. (Req 5.2.4)
   
    \item Test of device selection in device list.
   		\begin{precondition}
   			\item The MyDevices View is open
   			\item At least two devices are available
   			\item One device is selected
   		\end{precondition}
   		\begin{instruction}
   			\item Select an other device
   		\end{instruction}
   		\begin{postcondition}
   			\item Only the last selected device is still selected
   		\end{postcondition}
    %\item Only one device can be selected at a time. (Req. 5.2.5)
    
    \item Test of error message in device list.
    	\begin{precondition}
    		\item The MyDevices View is open
    		\item No device is selected
    	\end{precondition}
    	\begin{instruction}
    		\item Press the "Control Device"-button
    	\end{instruction}
    	\begin{postcondition}
    		\item A pop-up message "Please select a device" is displayed
    	\end{postcondition}
    %\item When no device is selected and the "Control Device" -button is pressed, a pop-up message "Please select a device" is displayed. (Req. 5.2.6)
   \newpage
	\item Test of device naming. \label{MyDevicesTest2}
		\begin{precondition}
			\item The MyDevices View is open
			\item There is at least one sensor in the list of available devices
			\item There is at least one light bulb in the list of available devices
		\end{precondition}
		\begin{instruction}
			\item Check the name of each device
			\item Check the address of each device
			\item Check the id of each device
		\end{instruction}
		\begin{postcondition}
			\item The name of each sensor device is "Sensor"
			\item The name of each light bulb is "Light Bulb"
			\item The address of each device is their MAC address
			\item The id of each device is their identifier
		\end{postcondition}
		   
    %\item Sensors are displayed in the list of available devices with "Sensor" as name, and it's MAC address as address and their indentifier as id. (Req. 5.2.7)
   
	\item Test of naming of light bulbs. \\
		Instructions for this test have been incorporated into \ref{MyDevicesTest2}

		   
    %\item Light bulbs are displayed in the list of available devices with "Light Bulb" as name and it's MAC address as address and their indentifier as id. (Req. 5.2.8)
   
	\item Test of scanning function.
		\begin{precondition}
			\item The MyDevices View is open
			\item There is at least one available device within range of the MVD
			\item The list of available devices is empty
		\end{precondition}
		\begin{instruction}
			\item Press the "Get Devices"-button
		\end{instruction}
		\begin{postcondition}
			\item A scan for available devices has been performed
			\item The list of available devices is updated
		\end{postcondition}
		   
    %\item The "Get Devices" -button performs a scan for available devices when pressed. (Req. 5.2.9)
   
	\item Test of termination of application.
		\begin{precondition}
			\item The MyDevices View is open
		\end{precondition}
		\begin{instruction}
			\item Press the back button
		\end{instruction}
		\begin{postcondition}
			\item The application is closed
		\end{postcondition}
		   
   %\item When the back button is pressed the application is closed (Req. 5.2.10)
    
    % Nytt requirement!
    \item Test if layout is consistent with specifications.
		\begin{precondition}
			\item The MyDevices View is open
		\end{precondition}
		
		\begin{instruction}
			\item Identify the "Get Devices"-button
			\item Identify the "Control Device"-button
			\item Identify the list of available devices
			\item Check the labels at the top of the list	
		\end{instruction}		
		
		\begin{postcondition}
 			\item The "Get Devices"-button is at the top of the view
 			\item The "Control Device"-button is at the bottom of the view
 			\item The list of available devices is in the middle of the view
 			\item The "ID"-label is placed at the top left of the list
 			\item The "Address"-label is placed at the top right of the list 
 			\item The "Device name"-label is placed inbetween the "Address"- and the "ID"-label. 
		\end{postcondition}
		   
    %\item The layout of the MyDevices View resemble figure 1 in appendix  Ref \ref{refdocs:srs} (5.2.11).
    
	\item Test of MAC addresses
		\begin{precondition}
			\item The "MyDevices View" is open
			\item The devices with the MAC addresses in Ref \ref{refdocs:srs} are within range of the MVD
			\item At least one device with an unknown MAC address is within range of the MVD
			\item All devices are active
		\end{precondition}
		
		\begin{instruction}
			\item Press the "Get Devices"-button
		\end{instruction}
		
		\begin{postcondition}
			\item The two devices with the MAC addresses in Ref \ref{refdocs:srs} are available
			\item No other devices are available
		\end{postcondition}
    
    %Test A.1.12: The "Get Devices" button only displays the devices with the MAC-addresses in Ref \ref{refdocs:srs} (Req. 5.2.12, 5.2.13)
\end{testlist}

\newpage
%--------------------------Sensor view tests --------------------------------------
\subsection{Sensor View Test Instructions}
\begin{testlist}
	\item Test of transition from main view to sensor view. \label{SensorTest1}
 		\begin{precondition}
 			\item The MyDevices View is open
 			\item A sensor device is available in the list
 		\end{precondition}
 		\begin{instruction}
 			\item Select a sensor device
 			\item Press the "Control Device"-button
 		\end{instruction}
 		\begin{postcondition}
 			\item The Sensor View is open
 			\item The sensor name is shown in the top of the view
 			\item The MAC address is shown in the top of the view
 			\item The temperature, pressure, humidity, magnetic field strength, gyroscopic data and acceleration text fields are all empty
 		\end{postcondition}

%When the "Control Device"-button in the MyDevices View is pressed and a sensor is selected, the Sensor view is opened (Req. 5.3.1).

	\item Test of naming of sensor. \\
		Instructions for this test have been incorporated into \ref{SensorTest1}
		
%\item The sensor name and mac-address is shown in the top of the view (Req. 5.3.2, 5.3.3).

	\item Test of on/off switch.
		\begin{precondition}
			\item The Sensor View is open
			\item The on/off-status switch is set to off
		\end{precondition}
		\begin{instruction}
			\item Set the on/off-status switch to on
	  	\end{instruction}
	  	\begin{postcondition}
	  		\item The on/off-status switch of the selected sensor is on
	  	\end{postcondition}
%\item It is possible to change the on/off -status of the selected sensor with a switch (Req. 5.3.4).

	\item Test of text field naming in Sensor View.
		\begin{precondition}
			\item The Sensor View is open
		\end{precondition}
		
		\begin{instruction}
			\item Identify the six text fields
			\item Check the labels preceding the six text fields
		\end{instruction}		
		
		\begin{postcondition}
			\item The labels preceding the six text fields are "T", "P", "H", "M", "G" and "A" respectively
		\end{postcondition}
%\item There are text fields preceded by "T", "P", "H", "M", "G", "A" that are used to, respectively, display temperature, pressure, humidity, magnetic field strength, gyroscopic data and acceleration (Req. 5.3.5, 5.3.6, 5.3.7, 5.3.8, 5.3.9, 5.3.10).

	\item Test of get-button functionality.
		\newline This test should be run once for each of the text fields and its corresponding physical observables.
		\begin{precondition}
			\item The Sensor View is open
			\item The on/off-status switch is set to on
			\item The sensor is in a controlled environment
		\end{precondition}
		\begin{instruction}
			\item Get a reference value for the environmental variable
			\item Press the corresponding "Get"-button to the text field
			\item Confirm that the reference value is consistent with the controlled environment
			\item Change the variable in the controlled environment
			\item Press the "Get"-button
		\end{instruction}
		\begin{postcondition}
			\item The values of the corresponding sensors are retrieved if available and displayed
			\item The retrieved value in step five is different from that in step two
		\end{postcondition}
%\item By pressing the corresponding "Get"-button, the values of the temperature, pressure, humidity, magnetic field strength, qyroscopic and acceleration sensors are retrieved if available and displayed (Req. 5.3.11, 5.3.12, 5.3.13, 5.3.14, 5.3.15, 5.3.16).

	\item Test of "Get All"-button functionality.
		\begin{precondition}
			\item The Sensor View is open
			\item The on/off-status switch is set to on
			\item The sensor is in a controlled environment
		\end{precondition}
		\begin{instruction}
			\item Get a reference value for one or more of the environmental variables
			\item Press the "Get All"-button
			\item Confirm that sensor data is consistent with the controlled environment
			\item Change one or more of the reference values in the control environment
			\item Press the "Get All"-button
		\end{instruction}
		\begin{postcondition}
			\item The new values for all six sensor's data are displayed in their corresponding text fields
			\item The retrieved values in step five are different from those in step two
		\end{postcondition}
%\item The "Get All" -button gets the values for all six sensors and displays them (Req. 5.3.17).

\newpage

	\item Test of "Clear All"-button functionality.
		\begin{precondition}
			\item The Sensor View is open
			\item Sensor data are displayed in the text fields
		\end{precondition}
		\begin{instruction}
			\item Press the "Clear All"-button
		\end{instruction}
		\begin{postcondition}
			\item All the sensor text fields are empty
		\end{postcondition}
%\item The "Clear All" button clears all sensor text fields from data (Req. 5.3.18).

	\item Test of unavailable data. \newline This test should be run 1 time for each of the text fields
		\begin{precondition}
			\item The Sensor View is open
			\item The on/off-status switch is set to on
			\item Some sensor data is unavailable
		\end{precondition}
		\begin{instruction}
			\item Press the corresponding "Get"-button to the text field
    	\end{instruction}
    	\begin{postcondition}
    		\item A pop-up message with the text "Error occurred" is displayed
    	\end{postcondition}
%\item If there is no data to retrieve for any of the physical quantities: temperature, pressure, humidity, magnetic field strength, gyroscopic data and acceleration when the corresponding "Get" button is pressed the corresponding, text fields display "No data available" (Req. 5.3.19, 5.3.20, 5.3.21, 5.3.22, 5.3.23, 5.3.24).

	\item Test of on/off-status switch functionality.
		\begin{precondition}
			\item The Sensor View is open
		\end{precondition}
		\begin{instruction}
			\item Change the status of the on/off-status switch
		\end{instruction}
		\begin{postcondition}
			\item The on/off-status switch corresponds to the status sent from the REST API.
		\end{postcondition}

%\item The on/off-switch is set according to the information from the REST API (Req. 5.3.25).

	\item Test of back button functionality in the Sensor View.
		\begin{precondition}
			\item The Sensor View is open
		\end{precondition}
		\begin{instruction}
			\item Press the back button
		\end{instruction}
		\begin{postcondition}
			\item The MyDevices View is open
		\end{postcondition}

%\item When the back button is pressed in the Sensor View, the MyDevices View is opened (Req. 5.3.26).

	\item Test of text fields on selecting new device. \\
		Instructions for this test have been incorporated into \ref{SensorTest1}

%\item When the sensor view is opened the temperature, pressure, humidity, magnetic field strength, gyroscopic data and acceleration text fields are empty (Req. 5.3.27).

	\item Test of resemblance of Sensor View to specification.
		\begin{precondition}
			\item The Sensor View is open
		\end{precondition}
		\begin{instruction}
			\item Identify the text field showing the name and the MAC address
			\item Identify the on/off-status switch
			\item Identify the six text fields and their corresponding "Get"-buttons
			\item Identify the "Get All"-button
			\item Identify the "Clear All"-button
		\end{instruction}
		\begin{postcondition}
			\item The text field showing the name and the MAC address is placed at the top of the view
			\item The on/off-status switch is placed below the text field described above
			\item The six text fields are aligned
			\item The six text fields are placed below the on/off-status switch
			\item The "Get"-buttons are aligned vertically
			\item The "Get"-buttons are aligned with their respective text fields
			\item The "Get All"-button is placed at the bottom left of the view
			\item The "Clear All"-button is placed at the bottom right of the view
			\item The "Get All"- and the "Clear All"-button are aligned horizontally
		\end{postcondition}
		
	\item Test of current data
		\begin{precondition}
			\item The Sensor View is open
			\item The sensor device has been turned off for at least ten minutes
			\item The text fields are empty
		\end{precondition}
		\begin{instruction}
			\item Press the "Get"-button corresponding to each text field
			\item Press the "Get All"-button
			\item Check the pop-up message.
		\end{instruction}
		\begin{postcondition}
			\item A pop-up message with the text "Error occurred" is displayed
		\end{postcondition}
\end{testlist}

\subsection{Light Bulb View Test Instructions}
\begin{testlist}

    \item Test of transition from the MyDevices View to the Light Bulb View. \label{LightBulbTest1}
    	\begin{precondition}
    		\item The MyDevices View is open
    		\item At least one light bulb is in the list of available devices
    	\end{precondition}
    	\begin{instruction}
    			\item Choose a light bulb in the list of available devices
    			\item Press the "Control Device"-button
    	\end{instruction}
    	\begin{postcondition}
    		\item The Light Bulb View is open
    		\item The name of the light bulb is shown in the top of the view
    		\item The MAC address of the light bulb is shown at the top of the view
    		\item The text fields specifying color are empty
    	\end{postcondition}

	%\item The Light Bulb View opens when a light bulb is chosen in the MyDevices View and the "Control device" button is pressed. (Req 5.4.1)
	
	\item Test of naming of light bulb.\\
		Instructions for this test have been incorporated into \ref{LightBulbTest1}
    	
	%\item The name of the selected light bulb is shown at the top of the View (Req 5.4.2).
	
	\item Test of light emittance from lamp.
		\begin{precondition}
			\item The Light Bulb View is open
			\item The on/off-status switch is set to on
			\item The state of the light bulb corresponds to the state of the switch
		\end{precondition}
    	\begin{instruction}
    		\item Change the on/off-status switch to off
    	\end{instruction}
    	\begin{postcondition}
    		\item The light bulb is off
    	\end{postcondition}  		
  
    % \item The selected light bulb can be turned on/off with a switch (Req. 5.4.3). 
   
   	\item Test of editability of text fields in Light Bulb View.
   		\begin{precondition}
   			\item The Light Bulb View is open
		\end{precondition}
    	\begin{instruction}
    		\item For each of the four fields, enter "A"
    	\end{instruction}
    	\begin{postcondition}
    		\item The four fields are preceded by "R:", "G:", "B:" and "W:" respectively
    		\item It is possible to enter a character into the fields
    	\end{postcondition}

    % \item The field of R-, G-, B-, W-value is editable and preceded by "R:", "G:", "B:" and "W:" respectively (Req. 5.4.4, 5.4.5, 5.4.6, 5.4.7).

	\item Test of starting values of text fields.\\
		Instructions for this test have been incorporated into \ref{LightBulbTest1}  
%	\item When the Light Bulb View is opened, the fields are empty (Req. 5.4.8).

	\item Test of "Get"-button functionality.
		\begin{precondition}
			\item The Light Bulb View is open
			\item The light bulb is on
			\item The light bulb has known values, set via other means than the application.
		\end{precondition}
		\begin{instruction}
			\item Press the "Get"-button
			\item Verify that the values are consistent with the values entered at precondition.
		\end{instruction}
		\begin{postcondition}
			\item The R-, G-, B-, W-values are displayed in the fields specified in Req. 5.4.4-5.4.7 in ref \ref{refdocs:srs}
		\end{postcondition}
%	\item The "Get"-button retrieves the R-, G-, B-, W-values and present them in their corresponding fields (Req. 5.4.9). 

\newpage

	\item Test of changeability of color of light bulb.
		\begin{config}[six]
			\item A=FF, B=00, C=00, D=00 (color: Red)
			\item A=00, B=FF, C=00, D=00 (color: Green)
			\item A=00, B=00, C=FF, D=00 (color: Blue)
			\item A=00, B=00, C=00, D=FF (color: White)
			\item A=F0, B=F0, C=F0, D=F0 (color: White)
			\item A=ff, B=ff, C=ff, D=ff (color: White)
		\end{config}
		\begin{precondition}
			\item The Light Bulb View is open
			\item The light bulb is on
		\end{precondition}
		\begin{instruction}
			\item Set the "R:"-field to A
			\item Set the "G:"-field to B
			\item Set the "B:"-field to C
			\item Set the "W:"-field to D
			\item Press the "Set"-button
		\end{instruction}
		\begin{postcondition}
			\item The light bulb has the specified color
		\end{postcondition}
		
%	\item The "Set button" sets the color of the light bulb (Req. 5.4.10).

	\item Test of interpretation of empty text fields.
    	\begin{config}[four]
			\item A blank, B=FF, C=00, D=00 (color: Green)
			\item A=00, B blank, C=FF, D=00 (color: Blue)
			\item A=00, B=00, C blank, D=FF (color: White)
			\item A=FF, B=00, C=00, D blank (color: Red)
		\end{config}
		\begin{precondition}
			\item The Light Bulb View is open
			\item The light bulb is on
			\item The R-, G-, B-, W-, fields show FF, FF, FF and FF respectively
			\item The light bulb glows white
		\end{precondition}
		\begin{instruction}
			\item Set the "R:"-field to A
			\item Set the "G:"-field to B
			\item Set the "B:"-field to C
			\item Set the "W:"-field to D
			\item Press the "Set"-button
			\item Press the "Get"-button
		\end{instruction}
		\begin{postcondition}
			\item The light bulb has the specified color
			\item The fields that were left blank show "00"
		\end{postcondition}
		
%	\item If an input value is left blank the value is interpreted as 00 (Req. 5.4.11) .   

	\item Test of allowed values in text fields.
		\begin{config}[three]
			\item A = '111'
			\item A = 'GG' 
			\item A = -1
		\end{config}
		\begin{precondition}
			\item The Light Bulb View is open
			\item The light bulb is on
		\end{precondition}
		\begin{instruction}
			\item Try to enter A into the "R:"-field
			\item Try to enter A into the "G:"-field
			\item Try to enter A into the "B:"-field
			\item Try to enter A into the "W:"-field
		\end{instruction}
		\begin{postcondition}
			\item The value A is not accepted
		\end{postcondition}
%    \item The fields only accepts two characters that represent hexadecimal numbers (e.g. 00 to FF and all combinations inbetween) (Req 5.4.12, Req 5.4.13).
\newpage

	\item Test of user feedback when color is changed.
		\begin{precondition}
			\item The Light Bulb View is open
			\item The light bulb is on
			\item The light bulb glows white
		\end{precondition}
		\begin{instruction}
			\item Set the "R:"-field to FF
			\item Set the "G:"-field to 00
			\item Set the "B:"-field to 00
			\item Set the "W:"-field to 00
			\item Press the "Set"-button
		\end{instruction}
		\begin{postcondition}
			\item A pop-up message saying "Color successfully changed" is displayed
			\item The light bulb is red
		\end{postcondition}
%    \item A pop-up message saying "Color successfully changed" is displayed when the values of the light bulb were successfully set. (Req 5.4.14)

	\item Test of user feedback when color could not be changed.
		\begin{precondition}
			\item The Light Bulb View is open
			\item The light bulb is on
			\item The light bulb glows red
			\item The color of the light bulb can not be changed
		\end{precondition}
		\begin{instruction}
			\item Set the "R:"-field to 00
			\item Set the "G:"-field to 00
			\item Set the "B:"-field to FF
			\item Set the "W:"-field to 00
			\item Press the "Set"-button
		\end{instruction}
		\begin{postcondition}
			\item A pop-up message with the text "Error occurred" is displayed
			\item The light bulb glows red
		\end{postcondition}
%    \item A pop-up message saying "Error: Could not change color." is displayed when the values of the light bulb were unsuccessfully set. (Req 5.4.15)
\newpage
	\item Test of set button characteristics.
		\begin{precondition}
			\item The Light Bulb View is open
			\item The light bulb is off
		\end{precondition}
		
		\begin{instruction}
			\item Identify the "Set"-button
		\end{instruction}

		\begin{postcondition}
			\item The "Set"-button is unavailable
		\end{postcondition}
%    \item When the light bulb is off, the "Set"-button is unavailable (Req. 5.4.16).

	\item Test of back button functionality in the Light Bulb View.
		\begin{precondition}
			\item The Light Bulb View is open
		\end{precondition}
		\begin{instruction}
			\item Press the back button
		\end{instruction}
		\begin{postcondition}
			\item The MyDevices View is open
		\end{postcondition}
%    \item When the back button is pressed the system switches to MyDevices View (Req. 5.4.17).
    
	\item Test if layout is consistent with specifications.
		\begin{precondition}
			\item The Light Bulb View is open
		\end{precondition}
		\begin{instruction}
			\item Identify the text field showing the name and the MAC address
			\item Identify the on/off-status switch
			\item Identify the four text fields
			\item Identify the "Get"-button
			\item Identify the "Set"-button
		\end{instruction}
		\begin{postcondition}
			\item The text field showing the name and the MAC address is placed at the top of the view
			\item The on/off-status switch is placed below the text field described above
			\item The four text fields are aligned
			\item The four text fields are placed below the on/off-status switch
			\item The "Get"-button is placed at the bottom left of the view
			\item The "Set"-button is placed at the bottom right of the view
			\item The "Get"- and the "Set"-button are aligned horizontally
		\end{postcondition}
%    \item The layout of the Light Bulb View resemble figure 3 in appendix  Ref \ref{refdocs:srs} (Req. 5.4.18).

\end{testlist}	

%--------------------Appendix B: System Test Specification ----------------
\newpage

\section{System Test Instruction} \label{appendix:section:systemtest}
This is appendix \ref{appendix:section:systemtest} where we list all system test specifications.

\subsection{Use Cases}
\begin{testlist}
	\item Test of Scenario 5.1.1.
		\begin{precondition}
			\item The MyDevices View is open
			\item There is a light bulb and a sensor device with MAC addresses as specified in Ref \ref{refdocs:srs} within scan range of the MVD
			\item No other known devices are within range of the MVD
		\end{precondition}
		\begin{instruction}
			\item Press the "Get Devices"-button
		\end{instruction}
		\begin{postcondition}
			\item The light bulb is displayed in the MyDevices View
		 	\item The sensor device is displayed in the MyDevices View
		\end{postcondition}
	%SVVS: Scenario 5.1.1 in Ref \ref{refdocs:srs} is supported (Req. 5.1.1).

	\item Test of exception handling in Scenario 5.1.1. \\
	This test has been redacted.
%		\begin{precondition}
%			\item The MyDevices View is open
%			\item No known devices are within range of the MVD
%		\end{precondition}
%		\begin{instruction}
%			\item Press the "Get Devices"-button
%		\end{instruction}
%		\begin{postcondition}
%			\item A pop-up message with the text "No devices found" is found
%			\item The list of avilable devices is empty
%		\end{postcondition}
	%SVVS: Exception 1 described in scenario 5.1.1 can be generated by removing the devices from range while attempting scenario 5.1.1 (Req 5.1.1).
	
	\item Test of Scenario 5.1.2. 
		\begin{precondition}
			\item The MyDevices View is open
			\item A light bulb is available
		\end{precondition}
		\begin{instruction}
			\item Select the light bulb
			\item Press the "Control Device"-button
		\end{instruction}
		\begin{postcondition}
			\item The Light Bulb View is open
		\end{postcondition}
%	\item Scenario 5.1.2 in Ref \ref{refdocs:srs} is supported (Req. 5.1.2).

	\item Test exception 1 in Scenario 5.1.2. \label{SystemTest1}
		\begin{precondition}
			\item The MyDevices View is open
			\item A light bulb and a sensor device is available
		\end{precondition}
		\begin{instruction}
			\item Press the "Control Device"-button
		\end{instruction}
		\begin{postcondition}
			\item A pop-up message with the text "Please select a device" is shown
		\end{postcondition}	
%	\item Exception 1 in scenario 5.1.2 can be generated by not selecting a device in step 1 (Req. 5.1.2).

	\item Test of Scenario 5.1.3.
		\begin{precondition}
			\item The MyDevices View is open
			\item A sensor device is available
		\end{precondition}
		\begin{instruction}
			\item Select the sensor
			\item Press the "Control Device"-button
		\end{instruction}
		\begin{postcondition}
			\item The Sensor View is open
		\end{postcondition}
%	\item Scenario 5.1.3 in Ref \ref{refdocs:srs} is supported (Req. 5.1.3).


	\item  Test exception 1 in Scenario 5.1.3. \\
		Instructions for this test have been incorporated into \ref{SystemTest1}
%	\item Exception 1 described in scenario 5.1.3 can be generated by not selecting a device in step 1 (Req. 5.1.3). 

	\item  Test of Scenario 5.1.4.
		\begin{precondition}
			\item The Light Bulb View is open
			\item The light bulb is on
		\end{precondition}
		\begin{instruction}
			\item Set the on/off-status switch to off
		\end{instruction}
		\begin{postcondition}
			\item The light bulb is off
		\end{postcondition}
%	\item Scenario 5.1.4 in Ref \ref{refdocs:srs} is supported (Req. 5.1.4).
\newpage
	\item  Test of Scenario 5.1.5.
		\begin{precondition}
			\item The Light Bulb View is open
			\item The light bulb is off
		\end{precondition}
		\begin{instruction}
			\item Set the on/off-status switch to on
		\end{instruction}
		\begin{postcondition}
			\item The light bulb is on
		\end{postcondition}
%	\item Scenario 5.1.5 in Ref \ref{refdocs:srs} is supported (Req. 5.1.5).

	\item  Test of Scenario 5.1.6.
		\begin{precondition}
			\item The Sensor View is open
			\item The sensor device is on
		\end{precondition}
		\begin{instruction}
			\item Set the on/off-status switch to off
		\end{instruction}
		\begin{postcondition}
			\item The sensor device is off
		\end{postcondition}
%	\item Scenario 5.1.6 in Ref \ref{refdocs:srs} is supported (Req. 5.1.6).

	\item Test of Scenario 5.1.7.
		\begin{precondition}
			\item The Sensor View is open
			\item The sensor device is off
		\end{precondition}
		\begin{instruction}
			\item Set the on/off-status switch to on
		\end{instruction}
		\begin{postcondition}
			\item The sensor device is on
		\end{postcondition}
%	\item Scenario 5.1.7 in Ref \ref{refdocs:srs} is supported (Req. 5.1.7).
\newpage
	\item  Test of Scenario 5.1.8.
		\begin{precondition}
			\item The Light Bulb View is open
			\item The light bulb is on
			\item The "R:"-field of the light bulb is set to FF
			\item The "G:"-field of the light bulb is set to FF
			\item The "B:"-field of the light bulb is set to 00
			\item The "W:"-field of the light bulb is set to 00
		\end{precondition}
		\begin{instruction}
			\item Set the "R:"-field to FF
			\item Set the "G:"-field to 00
			\item Set the "B:"-field to 00
			\item Set the "W:"-field to 00
			\item Press the "Set"-button
		\end{instruction}
		\begin{postcondition}
			\item The light bulb glows red
		\end{postcondition}
%	\item Scenario 5.1.8 in ref \ref{refdocs:srs} is supported (Req. 5.1.8).

	\item  Test of Scenario 5.1.9.
		\begin{precondition}
			\item The Light Bulb View is open
			\item The light bulb is on
			\item The light bulb glows yellow
			\item The R-value of the light bulb is FF
			\item The G-value of the light bulb is FF
			\item The B-value of the light bulb is 00
			\item The W-value of the light bulb is 00
		\end{precondition}
		\begin{instruction}
			\item Press the "Get"-button
		\end{instruction}
		\begin{postcondition}
			\item The "R:"-field displays FF
			\item The "G:"-field displays FF
			\item The "B:"-field displays 00
			\item The "W:"-field displays 00
		\end{postcondition}
%	\item Scenario 5.1.9 in Ref \ref{refdocs:srs} is supported (Req. 5.1.9).
\newpage
	\item  Test if no data is available in Scenario 5.1.9.
		\begin{precondition}
			\item The Light Bulb View is open
			\item The light bulb is on
			\item The light bulb glows yellow
			\item One or more R-, G-, B-, W-values are unavailable
		\end{precondition}
		\begin{instruction}
			\item Press the "Get"-button
			\item Confirm the pop-up message
		\end{instruction}
		\begin{postcondition}
			\item pop-up message with the text "Error occurred" is displayed
		\end{postcondition}
%	\item If some data is unavailable when attempting scenario 5.1.9, the text in the corresponding field is set to "No data available." (Req. 5.1.9).

	\item  Test of Scenario 5.1.10.
		\begin{precondition}
			\item The Sensor Device View is open
			\item The sensor device is on
		\end{precondition}
		\begin{instruction}
			\item Press the "Get"-button next to the "T:"-field
		\end{instruction}
		\begin{postcondition}
			\item The temperature is displayed in the "T:"-field
		\end{postcondition}
%	\item Scenario 5.1.10 in Ref \ref{refdocs:srs} is supported (Req. 5.1.10).

	\item  Test of Scenario 5.1.10, when some data is unavailable.
		\begin{precondition}
			\item The Sensor Device View is open
			\item The sensor device is on
			\item The temperature data for the sensor device is unavailable
		\end{precondition}
		\begin{instruction}
			\item Press the "Get"-button next to the "T:"-field
		\end{instruction}
		\begin{postcondition}
			\item A pop-up message with the text "Error occurred" is displayed
		\end{postcondition}
        
%	\item If some data is unavailable when attempting scenario 5.1.10, the text in the corresponding field is set to "No data available." (Req. 5.1.10).
\newpage
	\item  Test of Scenario 5.1.11.
		\begin{precondition} 
			\item The Sensor Device View is open
			\item The sensor device is on
		\end{precondition}
		\begin{instruction}
			\item Press the "Get All"-button at the bottom of the view
		\end{instruction}
		\begin{postcondition}
			\item All the sensor values are displayed in their respective field
		\end{postcondition}
%	\item Scenario 5.1.11 in Ref \ref{refdocs:srs} is supported (Req. 5.1.11).

	\item  Test of Scenario 5.1.11, when some data is unavailable.
		\begin{precondition}
			\item The Sensor Device View is open
			\item The sensor device is on
    	    \item Some sensor data for the sensor device is unavailable
		\end{precondition}
		\begin{instruction}
			\item Press the "Get All"-button
		\end{instruction}
		\begin{postcondition} 
			\item A pop-up message with the text "Error occurred" is displayed
		\end{postcondition}
%  		\item If some data is unavailable when attempting scenario 5.1.11, the text in the corresponding field is set to "No data available." (Req. 5.1.11).

	\item Test of Scenario 5.1.12.
		\begin{precondition} 
			\item The Sensor Device View is open
			\item At least one of the sensor data fields has data
		\end{precondition}
		\begin{instruction} 
			\item Press the "Clear All"-button
		\end{instruction}
	  	\begin{postcondition}
			\item All the sensor data fields are empty
		\end{postcondition}
%	\item Scenario 5.1.12 in Ref \ref{refdocs:srs} is supported (Req. 5.1.12).

	\item Test of exception 2 in scenario 5.1.2. 
		\begin{precondition}
			\item The MyDevices View is open
			\item A light bulb is available
		\end{precondition}
		\begin{instruction}
			\item Manually disconnect the light bulb
			\item Select the light bulb
			\item Press "Control Device"
		\end{instruction}
		\begin{postcondition}
			\item A pop-up message with the text "Device not found" is shown
		\end{postcondition}

	\item Test of exception 2 in scenario 5.1.3. 
		\begin{precondition}
			\item The MyDevices View is open
			\item A sensor device is available
		\end{precondition}
		\begin{instruction}
			\item Manually disconnect the sensor device
			\item Select the sensor device
			\item Press "Control Device"
		\end{instruction}
		\begin{postcondition}
			\item A pop-up message with the text "Device not found" is shown
		\end{postcondition}

	\item  Test of Exception 1 in Scenario 5.1.4.
		\begin{precondition}
			\item The Light Bulb View is open
			\item The light bulb is on
		\end{precondition}
		\begin{instruction}
			\item Disconnect the device running the app from the internet
			\item Set the on/off-status switch to off
		\end{instruction}
		\begin{postcondition}
			\item A pop-up message with the text "Error occurred" is displayed 
		\end{postcondition}
	
	\item  Test of Exception 1 in Scenario 5.1.5.
		\begin{precondition}
			\item The Light Bulb View is open
			\item The light bulb is off
		\end{precondition}
		\begin{instruction}
			\item Disconnect the device running the app from the internet
			\item Set the on/off-status switch to on
		\end{instruction}
		\begin{postcondition}
			\item A pop-up message with the text "Error occurred" is displayed
		\end{postcondition}	
	
	\item  Test of Exception 1 in Scenario 5.1.6.
		\begin{precondition}
			\item The Sensor View is open
			\item The sensor device is on
		\end{precondition}
		\begin{instruction}
			\item Disconnect the device running the app from the internet
			\item Set the on/off-status switch to off
		\end{instruction}
		\begin{postcondition}
			\item A pop-up message with the text "Error occurred" is displayed
		\end{postcondition}
		
	\item Test of Exception 1 in Scenario 5.1.7.
		\begin{precondition}
			\item The Sensor View is open
			\item The sensor device is off
		\end{precondition}
		\begin{instruction}
			\item Disconnect the device running the app from the internet
			\item Set the on/off-status switch to on
		\end{instruction}
		\begin{postcondition}
			\item A pop-up message with the text "Error occurred" is displayed
		\end{postcondition}
		
	\item  Test of Exception 1 in Scenario 5.1.8.
		\begin{precondition}
			\item The Light Bulb View is open
			\item The light bulb is on
			\item The "R:"-field of the light bulb is set to FF
			\item The "G:"-field of the light bulb is set to FF
			\item The "B:"-field of the light bulb is set to 00
			\item The "W:"-field of the light bulb is set to 00
		\end{precondition}
		\begin{instruction}
			\item Set the "R:"-field to FF
			\item Set the "G:"-field to 00
			\item Set the "B:"-field to 00
			\item Set the "W:"-field to 00
			\item Disconnect the device running the app from the internet
			\item Press the "Set"-button
		\end{instruction}
		\begin{postcondition}
			\item A pop-up message with the text "Error occurred" is displayed
		\end{postcondition}
		
\end{testlist}
\newpage
\subsection{Quality Test Instructions}

\begin{testlist}

	\item User friendlyness test \newline
		Measure the time it takes to perform this test
		\begin{precondition}
			\item 5 persons without prior knowledge of the project or the application are chosen
		\end{precondition}
		\begin{instruction}
			\item Give the test persons a minor introduction of what the application is able to do
            \item Tell the test persons to turn the lamp green
            \item Tell the test persons to collect all data from the sensor
		\end{instruction}
		\begin{postcondition}
			\item All test persons completed the tasks without any major problems
			\item The full test requires less than 10 minutes to perform
		\end{postcondition}
%	\item Test of usability using persons without prior knowledge of the project or the app (Req. 6.1).

\newpage
	\item Interaction time test. \newline
		For each step in the instruction below, measure the response time for the application. For the test to pass, the response time for each step should be less than 2 seconds.
		\begin{precondition}
			\item The application is not running
		\end{precondition}
		\begin{instruction}
			\item Start the application
            \item Press "Get Device"-button
            \item Select a sensor
            \item Press "Control Device"-button
            \item Turn the sensor on by flicking the switch
            \item Press "Get"-button. Do this for all data
            \item Press "Get All"-button
            \item Press "Clear All"-button
            \item Turn the sensor off by flicking the switch
            \item Press "Back"-button
            \item Select a light bulb
            \item Turn the ligt bulb on by flicking the switch
            \item Press "Get"-button
            \item Enter any value in the color fields
            \item Press "Set"-button
            \item Turn the light bulb off by flicking the switch
            \item Press "Back"-button until the application is closed
		\end{instruction}
		\begin{postcondition}
			\item Application is not running
		\end{postcondition}
%	\item The response time for any interaction with the system is below two seconds (Req. 6.2).
\newpage
	\item Test of response time for error messages.\newline
		For each step in the instruction below, measure the time until an error message is displayed. For the test to pass, the error message response time for each step should be 15 seconds.
		\begin{precondition} 
			\item In each step, the corresponding device should be out of range of the MVD
			\item In each step, first navigate to the appropriate view (some devices may need to be connected for this. In that case, remove them from range before the test instruction is performed)
		\end{precondition}
		\begin{instruction}
			\item Press the "Get Devices"-button
			\item Change the on/off-status switch of the sensor device to on
			\item Press the "Get All"-button in the Sensor Device View
			\item Change the on/off-status switch of the sensor device to off
			\item Change the on/off-status switch of the light bulb to on
			\item Change the color of the light bulb:
			\begin{itemize}
				\item Set the "R:"-field to 00
				\item Set the "G:"-field to 00
				\item Set the "B:"-field to FF
				\item Set the "W:"-field to 00
				\item Press the "Set"-button
			\end{itemize}
			\item Change the on/off-status switch of the light bulb to off
		\end{instruction}
%\item The time-out time for the system is 15 seconds (Req. 6.3). 
	
\end{testlist}

\clearpage

\end{appendices}


\end{document}