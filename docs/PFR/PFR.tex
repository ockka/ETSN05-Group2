\documentclass[a4paper]{article}
\usepackage[utf8]{inputenc}
\usepackage[english]{babel}
\usepackage{amsmath}
\usepackage{amsfonts}
\usepackage{amssymb}
\usepackage{graphicx}
\usepackage{geometry}
\usepackage{footnote}

\makesavenoteenv{tabular}


\title{PFR - Project Final Report}
\author{Team 2}

\begin{document}
\begin{titlepage}
\newgeometry{left=2cm,top=1cm,right=2cm}
\newcommand{\HRule}{\rule{\linewidth}{0.5mm}}

\begin{minipage}{0.5\textwidth}
\begin{flushleft} % Responsible persons, write on separate lines
\textit{Responsible for this document:}\\
Emma Albertz \\
Linnéa Claesson
\end{flushleft}
\end{minipage}
~
\begin{minipage}{0.4\textwidth}
\begin{flushright}
PUSS154219 v0.1 
\today
\end{flushright}
\end{minipage}\\[3cm]

\centering
\textsc{\LARGE Team 2}\\[0.5cm]

\HRule \\[0.4cm]
{ \huge \bfseries Project Final Report}\\[0.4cm] % Title of your document
\HRule \\[1.5cm]

\vfill
\begin{flushleft}
%Authors, write on separate lines
\textit{Authors of this document:}\\
Emma Albertz \\
Linnéa Claesson
\end{flushleft}



\end{titlepage}
\pagenumbering{gobble}



%\begin{center}
%\textit{\large Version History}
%
%    \begin{tabular}{ | l | l | l | p{5cm} |}
%    \hline
%    \textbf{Version} & \textbf{Date} & \textbf{Responsible} & \textbf{Description} \\ \hline
%    1.0 & 150916 & EA, LC & Baseline\\ \hline
%    \end{tabular}
%\end{center}



\setcounter{tocdepth}{2}
\tableofcontents
\newpage
\pagenumbering{arabic}

%-------------------------- Project Metrics --------------------------------------%
% Historisk överblick över projektet
% Vad är det som har hänt? Siffror, tabeller och diagram. Jämförelser och uppvisande av data. 
\section{Project Metrics}



%------------------------- Project Evaluation -----------------------------------%
% Utvärdering av vad som gick bra/dåligt
% Varför ser datan ut som den gör? Analys av de siffror man samlat in. Ex fasernas tidsåtgång, dokumentens antal timmar (UIFO), gruppernas timmar / vecka
\section{Project Evaluation}



%------------------------ Suggestions for Improvement ---------------------------%
% Förbättringsförslag
% Hur ska man göra för att förbättra de positiva trenderna och få bort de negativa? Processförbättring mm.
\section{Suggestions for Improvement}



%----------- Svar på frågeformuläret ---------------------------------%
\section{Svar på frågeformuläret}

%-------------------- PG -------------------------------------------%
\subsection{PG}
\subsubsection{Hur nöjd är du med ditt bidrag, varför?}
\subsubsection{Hur nöjd är du med din grupps prestation, varför?}
\subsubsection{Arbetsfördelning inom gruppen}
\subsubsection{Hur nöjd är du med produkten?}
\subsubsection{Vad har fungerat bra?}
\subsubsection{Vad har fungerat mindre bra?}
\subsubsection{Förbättringsförslag}
\subsubsection{Vetat vad och när du ska göra?}
\subsubsection{Tidsplan}
\subsubsection{Övriga kommentarer}

%------------------- SG ------------------------------------------%
\subsection{SG}
\subsubsection{Hur nöjd är du med ditt bidrag, varför?}
Antal timmar brukar väl ligga runt 10-12 timmar vilket är mindre än utsatt tid. O andra sidan är det inte många kurser jag lägger mer än den tiden. Tiden som har lags ned har varit det som har behövts.

Till kvalitet tycker jag att jag har gjort en mycket bra insats och vi inom gruppen har jobbat bra och kompletterar varandras åsikter väl. 
%---------

Jag har försökt efter bästa förmåga men emellanåt har jag inte varit riktigt nöjd med slutresultatet på vissa ställen i de olika rapporterna. Jag upplever att det är svårt att hålla en riktigt hög kvalitet när arbetet är så uppdelat och uppgiftsbeskrivningen otydlig. Det är helt enkelt svårt att få till en rödtråd genom alla dokumenten så att de når en riktigt hög kvalitet. Dock är kvaliteten överlag helt ok. För egen del borde jag kanske tagit en tydligare roll och haft bättre koll men det hade nog inte varit rimligt i förhållande till kursens storlek och ambitionen att få till en bra arbetsfördelning. Tidrapporteringen har jag misskött grovt när det gäller tid som lagts ner på enkla sysslor som diskussion via facebook. 
%---------



\subsubsection{Hur nöjd är du med din grupps prestation, varför?}
Det har fungerat jättebra, alla har presteras efter kraven och de har uppfyllts. 

\subsubsection{Arbetsfördelning inom gruppen}
Bra egentligen. Pga olika kompetensområden har det delats upp enligt det. Lite synd egentligen. Nu på senaste, då jag har varit den mest "tekniskt kunniga" har jag tagit fullt ansvar för det själv, vilket har varit en lite tyngre arbetsbörda att göra själv. Men allt som allt så har det delats upp det på ett hyfsat rättvist sätt och folk har inte dragit sig för att jobba. Jag tror att det har underlättat att vi bara är 3 personer och då "vågar" man inte gömma sig i mängden.

\subsubsection{Hur nöjd är du med produkten?}
Kan inte riktigt ge en dom än då jag inte testat den så mycket. Mest bara kollat i koden. Håller inte helt med vissa programmeringsbeslut men det är implementationsspecifikt och inte upp till mig. STLDD följs hyfsat bra ändå så egentligen är det inget fundamentalt att klaga på.

\subsubsection{Vad har fungerat bra?}
Strukturering och samarbete mellan grupper, de flesta har alltid varit jättevilliga att jobba och kommunicera. Är det något som är fel så är andra personer duktiga på att hoppa in och försöka hjälpa. Annars har det fungerat bra att ha ett möte varje vecka samma tid för att då vet man hur det fungerar varje vecka och man kan planera utifrån det.

\subsubsection{Vad har fungerat mindre bra?}
Android Studio självklart, hehe. Jag tyckte att det var tråkigt att det kändes som att UG inte försökte så mycket utan att man var tvungen att göra slides och alltihopa för att förklara saker. Jag menar, tänk så hade jag/vi i SG inte haft kompetens inom LaTeX/Git/Android. Vad hade hänt då? Det är ju bara ett sammanträffande att det råkar vara så. 

\subsubsection{Förbättringsförslag}
Till PG/gruppen: Egentligen ingenting, extremt välutfört projekt.

Till kursledningen: Fixa Android Studio för f.., bättre labbar (switch 'em up)

\subsubsection{Vetat vad och när du ska göra?}
Det finns alltid en mall att gå efter. Och man kan inte riktigt "göra fel", utan dfeft är mycket som är öppet för tolkning. Angående deadlines så har vi (ni kanske i PG iofs) alltid varit tydliga med det, vilket ger automatisk struktur i processen

\subsubsection{Tidsplan}
Blev ärligt talat förvånad över hur mycket tid som hade lagts ner på tidsplaneringen. Kan ju erkänna att det är rätt slappt, men dokumentsspecifika deadlines som kommer från kursledningen kan man juu inte ändra på. Annars har ju alla Gantt charts o planeringar o allt vad det varit varit superba!

\subsubsection{Övriga kommentarer}

%---------------------------- UG --------------------------------%

\subsection{UG}
\subsubsection{Hur nöjd är du med ditt bidrag, varför?}
De första veckorna hade jag lite att göra, och den kändes som om jag inte kunde bidra mycket till projektet. Efter att utvecklingen började har jag lagt ner mycket mer tid och känner mig därför ganska nöjd med mitt bidrag.
%-------

Jag har fått ta ansvar tillsammans med min "kodpartner" om ett visst segment av appen. Detta har lett till både frustration över problemen som dykt upp och stolthet över fungerande kod. Givet antalet timmar som jag lagt ner på projektet så är jag väldigt nöjd med vad jag producerat (trots problemen längs vägen) givet att jag aldrig kodat en Android app tidigare.
%------

\subsubsection{Hur nöjd är du med din grupps prestation, varför?}
Det mesta har gått ganska smärtfritt. I början av utvecklingen hade vissa svårt att komma igång för att de inte visste hur de skulle använda klasser som de inte själv skrev. Detta kan bero på att arbetet delades upp och kodningen började innan STLDD granskades så att när dokumentet ändrades var inte alla med på exakt vilka uppgifter som tillhörde de olika klasserna. Ett sätt att förhindra detta kan vara att låta UG börja med att göra ett litet system för att få känsla av vilka klasser som behövs och vad de behöver göra samtidigt som STLDD skrivs, så att UG kan lämna synpunkter på designen innan utveckling av den slutliga produkten börjar. På det sättet börjar UG:s arbete tidigare samtidigt som det kan bli lättare att komma igång när man börjar med ett mindre system.
%-----

Under hela projektet har jag haft känslan att alla driver på varandra för att projektet ska bli så bra som möjligt. Vi har hjälpt varandra och fört en kontinuerlig dialog under hela projektet. Under arbetets gång har vi ofta suttit tillsammans i en sal och kodat vilket gör det mycket enklare om man behöver fråga om något.
%-----



\subsubsection{Arbetsfördelning inom gruppen}
Uppdelningen skedde efter förslag från SG. Eftersom (någon i) SG hade erfarenhet av att programmera Android-appar hade de en uppfattning om ungefär hur lång tid varje del skulle ta och kunde göra en (enligt mig) bra uppdelning innan utvecklingen började. Att jobba i grupp om 2 personer har fungerat bra. De små grupperna gör att alla har ansvar för något vilket gör det lätt att bidra. Dessutom var de flesta i UG inte vana vid android-utveckling och att jobba i grupp om 2 personer gör det lättare att jobba med något som i början verkade nytt och svårt.
%-------

När vi fick UML-diagrammen från Systemgruppen så skickade de också med en föreslagen uppdelning av arbetet. Vi utgick från denna och delade in oss i fyra grupper om två och två, sen valde varje grupp vad de var intresserade av att koda och körde igång.
%-------



\subsubsection{Hur nöjd är du med produkten?}
En sak som gör mig mindre nöjd var att jag, speciellt i början, inte hade så bra koll på vad MVD och backend gör och hur vi kan använda det. Eftersom vi använder oss av dem hade det varit bra att ha tillgång till en lätt översikt om vad de gör som vi kan använda oss av i projektet eller att läsa igenom Technical System Description innan utvecklingen började, något jag inte gjorde. Eftersom vi inte visste om vad den användes till och vilka problem den hade ställdes hade vi krav eller förväntningar på produkten som vi inte har kunnat uppnå. Systemet är dock inte helt klart, så vissa problem kan fortfarande lösas.

En sak som jag är nöjd med är att efter uppdelningen inom UG så har alla grupper klarat sig ganska bra själva och klarat av interna deadlines. Samtidigt har kommunikationen inom UG fungerat bra vilket gör att jag känner mig ha bra koll på vad produkten klarar av och kan komma med kommentarer om det är något jag är missnöjd med.
%-----

Jag är väldigt nöjd med appen vi har producerat trots stora tekniska problem i kursens början, framförallt med Android Studio. Många av de problem som vi haft har i stort haft sitt ursprung i att vi var väldigt ovana vid Android programmering.
%-----


\subsubsection{Vad har fungerat bra?}
PG har varit bra på att hålla oss uppdaterade om vad som händer varje vecka.
Personer med erfarenhet (speciellt Daniel Olsson) har varit bra på att hjälpa de som inte har det.
SG har varit bra på att kommunicera med oss i UG om hur utvecklingen går framåt.
%------

I princip allt som har gjort har löpt väldigt smidigt, framförallt den öppna och flytande dialogen som förts under projektet har varit väldigt bra.
%-----


\subsubsection{Vad har fungerat mindre bra?}
Android Studio har inte fungerat på skoldatorerna. Vi trodde att det skulle finnas så vi fixade det inte i förväg till våra egna datorer, utan fick fixa det efter mötet där vi skulle börja med utvecklingen. På grund av detta startade vi några dagar senare än förväntat med kodningen.
%------

Kommunikationen med kursens personal verkar från mitt perspektiv ha haltat något. Jag har själv inte haft anledning att söka kontakt med "experterna" men har under möten hört om problem med detta.

När det gäller projektet så är det enda som fungerat mindre bra det faktum att skolans datorer inte har Android Studio.
%--------


\subsubsection{Förbättringsförslag}
Låta UG ha större ansvar för STLDD som jag skrev tidigare i formuläret.
Något kort om hur man kan skriva ett program som kommunicerar med databasen (hämta sensorvärden etc.). Detta kan ingå i en labb.

%-------
Jag har inga förslag på förbättringar.


\subsubsection{Vetat vad och när du ska göra?}
Som utvecklare var jag inte så delaktig de första veckorna av projektet och då kände jag att jag inte hade koll på vad jag var ansvarig för under hela projektet. Efter att vi kom igång inom UG känns det som om jag har bättre koll på vad jag behöver göra.

PG har varit tydliga med att berätta när de olika delarna ska vara klara. De interna deadlines som vi har satt inom UG har dock inte varit så strikta. Detta kan bero på att vi inte har erfarenhet av android-programmering och därför har svårt att uppskatta hur lång tid det kommer ta.
%-------

Våra veckovisa möten har varit till stor hjälp när det gäller att hålla sig uppdaterad om deadlines och vad som ska göras.
%-------


\subsubsection{Tidsplan}
Tidsplanen verkar överlag rimlig. Som utvecklare har jag bara varit ansvarig för delar av SRS och SDDD, så jag kan inte säga mycket om de andra dokumenten. En kommentar är som innan att det är svårt att börja koda medan STLDD fortfarande håller på att skrivas. Eftersom man vill följa designen får man ibland ändra i koden när designen uppdateras, vilket har skett några gånger. Det har också skett (mindre) problem med att när STLDD uppdateras kanske den som skriver en klass använder en nyare version än den som använder klassen, vilket skapar problem när de ska sammanfogas.

För SDDD (som inte är klar än) verkar det som om vi har fått lagom med tid.
%------

Det var lite snålt om tid i början när SRS:en och SVVS:en skulle produceras men utöver det så tycker jag planeringen varit väldigt bra.

\subsubsection{Övriga kommentarer}
Ett väl genomfört projekt trots vissa oklarheter från kursens personal har projekter fortsatt i rätt riktning. En bra känsla av att gruppens medlemmar är motiverade och vill göra projektet till något bra har definitivt bidragit mycket till resultatet.

%------------------------------- TG --------------------------------%
\subsection{TG}
\subsubsection{Hur nöjd är du med ditt bidrag, varför?}
\subsubsection{Hur nöjd är du med din grupps prestation, varför?}
\subsubsection{Arbetsfördelning inom gruppen}
\subsubsection{Hur nöjd är du med produkten?}
\subsubsection{Vad har fungerat bra?}
\subsubsection{Vad har fungerat mindre bra?}
\subsubsection{Förbättringsförslag}
\subsubsection{Vetat vad och när du ska göra?}
\subsubsection{Tidsplan}
\subsubsection{Övriga kommentarer}






\end{document}