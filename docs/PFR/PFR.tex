\documentclass[a4paper]{article}
\usepackage[utf8]{inputenc}
\usepackage[english]{babel}
\usepackage{amsmath}
\usepackage{amsfonts}
\usepackage{amssymb}
\usepackage{graphicx}
\usepackage{geometry}
\usepackage{footnote}

\makesavenoteenv{tabular}


\title{PFR - Project Final Report}
\author{Team 2}

\begin{document}
\begin{titlepage}
\newgeometry{left=2cm,top=1cm,right=2cm}
\newcommand{\HRule}{\rule{\linewidth}{0.5mm}}

\begin{minipage}{0.5\textwidth}
\begin{flushleft} % Responsible persons, write on separate lines
\textit{Responsible for this document:}\\
Emma Albertz \\
Linnéa Claesson
\end{flushleft}
\end{minipage}
~
\begin{minipage}{0.4\textwidth}
\begin{flushright}
PUSS154219 v0.1 
\today
\end{flushright}
\end{minipage}\\[3cm]

\centering
\textsc{\LARGE Team 2}\\[0.5cm]

\HRule \\[0.4cm]
{ \huge \bfseries Project Final Report}\\[0.4cm] % Title of your document
\HRule \\[1.5cm]

\vfill
\begin{flushleft}
%Authors, write on separate lines
\textit{Authors of this document:}\\
Emma Albertz \\
Linnéa Claesson
\end{flushleft}



\end{titlepage}
\pagenumbering{gobble}



%\begin{center}
%\textit{\large Version History}
%
%    \begin{tabular}{ | l | l | l | p{5cm} |}
%    \hline
%    \textbf{Version} & \textbf{Date} & \textbf{Responsible} & \textbf{Description} \\ \hline
%    1.0 & 150916 & EA, LC & Baseline\\ \hline
%    \end{tabular}
%\end{center}



\setcounter{tocdepth}{2}
\tableofcontents
\newpage
\pagenumbering{arabic}

%-------------------------- Project Metrics --------------------------------------%
% Historisk överblick över projektet
% Vad är det som har hänt? Siffror, tabeller och diagram. Jämförelser och uppvisande av data. 
\section{Project Metrics}



%------------------------- Project Evaluation -----------------------------------%
% Utvärdering av vad som gick bra/dåligt
% Varför ser datan ut som den gör? Analys av de siffror man samlat in. Ex fasernas tidsåtgång, dokumentens antal timmar (UIFO), gruppernas timmar / vecka
\section{Project Evaluation}



%------------------------ Suggestions for Improvement ---------------------------%
% Förbättringsförslag
% Hur ska man göra för att förbättra de positiva trenderna och få bort de negativa? Processförbättring mm.
\section{Suggestions for Improvement}



%----------- Svar på frågeformuläret ---------------------------------%
\section{Svar på frågeformuläret}

%-------------------- PG -------------------------------------------%
\subsection{PG}
\subsubsection{Hur nöjd är du med ditt bidrag, varför?}
\subsubsection{Hur nöjd är du med din grupps prestation, varför?}
\subsubsection{Arbetsfördelning inom gruppen}
\subsubsection{Hur nöjd är du med produkten?}
\subsubsection{Vad har fungerat bra?}
\subsubsection{Vad har fungerat mindre bra?}
\subsubsection{Förbättringsförslag}
\subsubsection{Vetat vad och när du ska göra?}
\subsubsection{Tidsplan}
\subsubsection{Övriga kommentarer}

%------------------- SG ------------------------------------------%
\subsection{SG}
\subsubsection{Hur nöjd är du med ditt bidrag, varför?}
Antal timmar brukar väl ligga runt 10-12 timmar vilket är mindre än utsatt tid. O andra sidan är det inte många kurser jag lägger mer än den tiden. Tiden som har lags ned har varit det som har behövts.

Till kvalitet tycker jag att jag har gjort en mycket bra insats och vi inom gruppen har jobbat bra och kompletterar varandras åsikter väl. 
%---------

% Gruppledare
Jag har försökt efter bästa förmåga men emellanåt har jag inte varit riktigt nöjd med slutresultatet på vissa ställen i de olika rapporterna. Jag upplever att det är svårt att hålla en riktigt hög kvalitet när arbetet är så uppdelat och uppgiftsbeskrivningen otydlig. Det är helt enkelt svårt att få till en rödtråd genom alla dokumenten så att de når en riktigt hög kvalitet. Dock är kvaliteten överlag helt ok. För egen del borde jag kanske tagit en tydligare roll och haft bättre koll men det hade nog inte varit rimligt i förhållande till kursens storlek och ambitionen att få till en bra arbetsfördelning. Tidrapporteringen har jag misskött grovt när det gäller tid som lagts ner på enkla sysslor som diskussion via facebook. 
%---------

Jag är mycket nöjd med mitt bidrag till projektet. Jag har tagit ansvar för att dokumenten ska bli klara i tid med bra kvalité. Responsen vi som grupp har fått har varit bra och det är ett gott tecken. Jag har även bidragit med mina tidigare erfarenheter till projektet vilket har varit positivt för gruppen.

\subsubsection{Hur nöjd är du med din grupps prestation, varför?}
Det har fungerat jättebra, alla har presteras efter kraven och de har uppfyllts. 
%-------

% Gruppledare
Om jag hade gjort det här på riktigt hade jag varit mycket tuffare vad det gäller kvalitetskrav på det vi släppt ifrån oss. Kompetensen i gruppen har varit hög och vi borde kunna prestera ännu bättre. Vi skulle kunna tagit ett större ansvar för kvaliteten genomgående och höjt ambitionsnivån för hela projektet.
%--------

Jag är mycket nöjd då vi har fått en struktur på projektet och gruppen där alla verkar nöjda. Dokumenten vi har haft ansvar för har fått bra respons där Alma och övriga projektmedlemmar är delaktiga. Jag tycker även att vi har lyckats styra upp projektet på ett bra sätt där UG har varit nöjda med STLDD för att kunna utveckla SDDD och även TG har varit nöjda med SRS. 

\subsubsection{Arbetsfördelning inom gruppen}
Bra egentligen. Pga olika kompetensområden har det delats upp enligt det. Lite synd egentligen. Nu på senaste, då jag har varit den mest "tekniskt kunniga" har jag tagit fullt ansvar för det själv, vilket har varit en lite tyngre arbetsbörda att göra själv. Men allt som allt så har det delats upp det på ett hyfsat rättvist sätt och folk har inte dragit sig för att jobba. Jag tror att det har underlättat att vi bara är 3 personer och då "vågar" man inte gömma sig i mängden.
%-------

% Gruppledare
Ambitionen från min sida var från start att inkludera övriga gruppmedlemmar fullt ut och i praktiken så tycker jag det har fungerat väldigt bra. Övriga gruppmedlemmar har tagit ansvar på egen hand för att vi gemensamt ska leverera det som förväntats av oss från andra grupper. Jag har svårt att se hur vi skulle kunna göra arbetsfördelningen speciellt mycket jämnare och jag tycker vi har utnyttjat de olika gruppmedlemmarnas styrkor. 
%----------

"Vi har på ett bra sätt strukturerat arbetet. Ofta träffats i skolan för att jobba tillsammans eller ha ett möte för att besluta vad som ska göras och se till att vi i gruppen är samstämmiga. 
Det har varit viktigt med samarbetet eftersom vi inte vill att man ska få olika svar beroende på vilka gruppmedlemmar man pratar med. Vi har därför ofta tagit ett snack internt innan vi går ut med ett svar. Det har varit väldigt bra då hela gruppen är med på vad som har skett och vilka åtgärder som ska vidtas.
Vi har även på ett bra sätt använt de kunskaper och erfarenheter vi tidigare samlat på oss för att bidra till projektet."

\subsubsection{Hur nöjd är du med produkten?}
Kan inte riktigt ge en dom än då jag inte testat den så mycket. Mest bara kollat i koden. Håller inte helt med vissa programmeringsbeslut men det är implementationsspecifikt och inte upp till mig. STLDD följs hyfsat bra ändå så egentligen är det inget fundamentalt att klaga på.
%-------

% Gruppledare
Svårt att svara på men jag litar på att UG gör ett bra jobb och ser i dagsläget ingen anledning till att det inte ska bli en bra produkt. Om inte resultatet är bra nog när det närmar sig deadline räknar jag med att vi har resurser och kompetens inom projektet för att lyckas få till ett bra resultat. Jag är dock inte helt övertygad om att lösningen är optimal och jag har överlag svårt att vara helt nöjd om det inte är perfekt.  
%---------

"Produkten är i dagens läge inte färdig, men jag anser att UG gör ett jättebra jobb och tror att den slutgiltiga produkten kommer vara något alla är nöjda med. UG har tagit ett bra ansvar efter att vi lämnade över STLDD, de har haft lite förfrågningar men det har vi i SG hjälpt till med så att de inte ska vara några problem.
Jag tror även att TG kommer hitta de fel som om möjligt finns och göra vår slutprodukt ännu bättre."

\subsubsection{Vad har fungerat bra?}
Strukturering och samarbete mellan grupper, de flesta har alltid varit jättevilliga att jobba och kommunicera. Är det något som är fel så är andra personer duktiga på att hoppa in och försöka hjälpa. Annars har det fungerat bra att ha ett möte varje vecka samma tid för att då vet man hur det fungerar varje vecka och man kan planera utifrån det.
%-------

% Gruppledare
Tempot har fungerat bra. Tycker också att det har fungerat bra med kommunikation i form av veckomöten.
%-----------

"Den struktur vi satte upp i projektet har fungerat bra och alla verkar vara nöjda. Alla har följt planeringen och än så länge har inte korthuset rasat.
Tycker även att årets projekt känns väldigt dagsaktuellt och det är kul att utveckla något som faktiskt skulle kunna säljas vidare ut till allmänheten."

\subsubsection{Vad har fungerat mindre bra?}
Android Studio självklart, hehe. Jag tyckte att det var tråkigt att det kändes som att UG inte försökte så mycket utan att man var tvungen att göra slides och alltihopa för att förklara saker. Jag menar, tänk så hade jag/vi i SG inte haft kompetens inom LaTeX/Git/Android. Vad hade hänt då? Det är ju bara ett sammanträffande att det råkar vara så. 
%--------

% Gruppledare
Kvaliteten har emellanåt blivit lidande på grund av det höga tempot även om kvaliteten fortfarande är på en hög nivå. I början på projektet hade det varit bra att tydligare strukturera upp ansvarsområden och komma bort från att kursledningens "röda bok" är lag och istället formera oss för att lösa problemet. Det fanns en del "problem" inom SG i början kring vad vår roll skulle vara och hur stort ansvar vi skulle ta.  Tycker också att projektet i sin helhet är för litet för antalet projektmedlemmar vilket ibland försämrar kvalitetsnivån i projektet. 
%------

"De krav kunden satt i början av projektet ändrades en tid in då vi redan hade satt flertalet dokument i baseline och viss del av koden var implementerad. Om detta var meningen från projektledningens håll för att lära sig hur det kan fungera ute i arbetslivet eller det bara är för att projektet är nytt är oklart. Jag tycker att man i början av kursen kan uppmärksamma studenterna om att specifikationen kan komma att ändras för att vi ska simulera detta om så är fallet. 
Specifikationen på applikationen lämnar ganska mycket till tolkning för läsaren möjligt att det kan förbättras till nästa år då detta är första året.
"

\subsubsection{Förbättringsförslag}
Till PG/gruppen: Egentligen ingenting, extremt välutfört projekt.

Till kursledningen: Fixa Android Studio för f.., bättre labbar (switch 'em up)
%--------

% Gruppledare
"Det känns också som att det har saknats en del kundkontakt för att försäkra oss om att vi faktiskt bygger rätt saker och där tycker jag att projektledningen kunde tagit en aktivare roll. 

Administrationen har varit omfattande och det hade varit bra om den kunde reducerats. 
%-------

"Bättre kontakt från vår sida med kunden för att diskutera det i specifikationen vi kände var otydligt. 
Att kursledningen skulle vara lite mer homogena i sina svar och inte ge olika beroende på vem man pratar med.
Bättre användning av piazza då det är en bra idé.
Android studio på skoldatorer."


\subsubsection{Vetat vad och när du ska göra?}
Det finns alltid en mall att gå efter. Och man kan inte riktigt "göra fel", utan dfeft är mycket som är öppet för tolkning. Angående deadlines så har vi (ni kanske i PG iofs) alltid varit tydliga med det, vilket ger automatisk struktur i processen
%-------

% Gruppledare
Otydligheterna har i regel introducerats genom kursansvariga i form av dåliga förberedelser eller illa fungerande programvara. Även de exakta kraven för prestationsbedömning och hur betygssättningen går till har varit otydligt. Däremot har det emellanåt varit en del tighta deadlines som försämrat resultatet. Det hade nog gått att kombinera tighta deadlines med en iterativ process för att både hålla tempot uppe och samtidigt höja kvaliteten.
%-------

Vi har på projektmötena varje vecka tydligt sagt en deadline på de olika leverabler vi har framför oss. Vilka som är ansvariga och när det ska levereras. Om man om möjligt inte kan hålla satt deadline har detta tagits upp.

\subsubsection{Tidsplan}
Blev ärligt talat förvånad över hur mycket tid som hade lagts ner på tidsplaneringen. Kan ju erkänna att det är rätt slappt, men dokumentsspecifika deadlines som kommer från kursledningen kan man juu inte ändra på. Annars har ju alla Gantt charts o planeringar o allt vad det varit varit superba!
%--------

% Gruppledare
Bra med högt tempo. Skulle dock önskat mer tid för att få en gemensam samsyn på hur en del dokument ska se ut och att höja prestationskraven. Framförallt finns det dokument som hade mått bra av en mer iterativ process.
%-------

Vi i gruppen kom överens om att vi ville vara klara med projektet innan tentaveckan. Med det bestämt visste vi att det skulle bli mycket att göra i början (för tex SG). Jag tycker att det har funkat bra och vi har varje vecka kontrollerat att tidsplanen håller och om inte så har man fått säga till. Men än så länge är den satta tidsplanen uppfylld och ingen i gruppen har uttryckt annat. 

\subsubsection{Övriga kommentarer}
%---- Gruppledare
Överlag har det fungerat bra. Bra jobbat!

%----------------------------------------------------------------%
%---------------------------- UG --------------------------------%
%----------------------------------------------------------------%

\subsection{UG}
\subsubsection{Hur nöjd är du med ditt bidrag, varför?}
De första veckorna hade jag lite att göra, och den kändes som om jag inte kunde bidra mycket till projektet. Efter att utvecklingen började har jag lagt ner mycket mer tid och känner mig därför ganska nöjd med mitt bidrag.
%-------

Jag har fått ta ansvar tillsammans med min "kodpartner" om ett visst segment av appen. Detta har lett till både frustration över problemen som dykt upp och stolthet över fungerande kod. Givet antalet timmar som jag lagt ner på projektet så är jag väldigt nöjd med vad jag producerat (trots problemen längs vägen) givet att jag aldrig kodat en Android app tidigare.
%------

Hade kunnat göra arbetet noggrannare och lagt mer tid för att lösa uppgiften på ett snyggare sätt, men har fortfarande varit delaktig i alla delar utav arbetet. 
%-------



\subsubsection{Hur nöjd är du med din grupps prestation, varför?}
Det mesta har gått ganska smärtfritt. I början av utvecklingen hade vissa svårt att komma igång för att de inte visste hur de skulle använda klasser som de inte själv skrev. Detta kan bero på att arbetet delades upp och kodningen började innan STLDD granskades så att när dokumentet ändrades var inte alla med på exakt vilka uppgifter som tillhörde de olika klasserna. Ett sätt att förhindra detta kan vara att låta UG börja med att göra ett litet system för att få känsla av vilka klasser som behövs och vad de behöver göra samtidigt som STLDD skrivs, så att UG kan lämna synpunkter på designen innan utveckling av den slutliga produkten börjar. På det sättet börjar UG:s arbete tidigare samtidigt som det kan bli lättare att komma igång när man börjar med ett mindre system.
%-----

Under hela projektet har jag haft känslan att alla driver på varandra för att projektet ska bli så bra som möjligt. Vi har hjälpt varandra och fört en kontinuerlig dialog under hela projektet. Under arbetets gång har vi ofta suttit tillsammans i en sal och kodat vilket gör det mycket enklare om man behöver fråga om något.
%-----

Vi har presterat tillräckligt bra men hade vi haft mer vilja hade vi kunnat prestera bättre. 

\subsubsection{Arbetsfördelning inom gruppen}
Uppdelningen skedde efter förslag från SG. Eftersom (någon i) SG hade erfarenhet av att programmera Android-appar hade de en uppfattning om ungefär hur lång tid varje del skulle ta och kunde göra en (enligt mig) bra uppdelning innan utvecklingen började. Att jobba i grupp om 2 personer har fungerat bra. De små grupperna gör att alla har ansvar för något vilket gör det lätt att bidra. Dessutom var de flesta i UG inte vana vid android-utveckling och att jobba i grupp om 2 personer gör det lättare att jobba med något som i början verkade nytt och svårt.
%-------

När vi fick UML-diagrammen från Systemgruppen så skickade de också med en föreslagen uppdelning av arbetet. Vi utgick från denna och delade in oss i fyra grupper om två och två, sen valde varje grupp vad de var intresserade av att koda och körde igång.
%-------

Alla i gruppen har varit delaktiga och gjort formella granskningar. Gruppens huvuduppgift SDDD delades upp i fyra ungefär likvärdigt stora delar som parvis löstes. Dock kan arbetsfördelningen inom paren ha varierat av olika anledningar.
%--------



\subsubsection{Hur nöjd är du med produkten?}
En sak som gör mig mindre nöjd var att jag, speciellt i början, inte hade så bra koll på vad MVD och backend gör och hur vi kan använda det. Eftersom vi använder oss av dem hade det varit bra att ha tillgång till en lätt översikt om vad de gör som vi kan använda oss av i projektet eller att läsa igenom Technical System Description innan utvecklingen började, något jag inte gjorde. Eftersom vi inte visste om vad den användes till och vilka problem den hade ställdes hade vi krav eller förväntningar på produkten som vi inte har kunnat uppnå. Systemet är dock inte helt klart, så vissa problem kan fortfarande lösas.

En sak som jag är nöjd med är att efter uppdelningen inom UG så har alla grupper klarat sig ganska bra själva och klarat av interna deadlines. Samtidigt har kommunikationen inom UG fungerat bra vilket gör att jag känner mig ha bra koll på vad produkten klarar av och kan komma med kommentarer om det är något jag är missnöjd med.
%-----

Jag är väldigt nöjd med appen vi har producerat trots stora tekniska problem i kursens början, framförallt med Android Studio. Många av de problem som vi haft har i stort haft sitt ursprung i att vi var väldigt ovana vid Android programmering.
%-----

Produkten fungerar och uppfyller kraven. Produktlösningen kunde ha gjorts på ett bättre sätt. 
%-------



\subsubsection{Vad har fungerat bra?}
PG har varit bra på att hålla oss uppdaterade om vad som händer varje vecka.
Personer med erfarenhet (speciellt Daniel Olsson) har varit bra på att hjälpa de som inte har det.
SG har varit bra på att kommunicera med oss i UG om hur utvecklingen går framåt.
%------

I princip allt som har gjort har löpt väldigt smidigt, framförallt den öppna och flytande dialogen som förts under projektet har varit väldigt bra.
%-----

Kommunikationen har fungerat och informationen har nått ut till alla. I utvecklingsgruppen så har de flesta varit delaktiga i diskussioner kring lösning samt eventuella ändringar. 

%-------

\subsubsection{Vad har fungerat mindre bra?}
Android Studio har inte fungerat på skoldatorerna. Vi trodde att det skulle finnas så vi fixade det inte i förväg till våra egna datorer, utan fick fixa det efter mötet där vi skulle börja med utvecklingen. På grund av detta startade vi några dagar senare än förväntat med kodningen.
%------

Kommunikationen med kursens personal verkar från mitt perspektiv ha haltat något. Jag har själv inte haft anledning att söka kontakt med "experterna" men har under möten hört om problem med detta.

När det gäller projektet så är det enda som fungerat mindre bra det faktum att skolans datorer inte har Android Studio.
%--------

"Utrustningen har varit bristande både med mjukvara till utveckling så som Android Studio på skoldatorer. Även hårdvaran med sensor och lampa har haft en del backend fel. 

En del av arbetet har haft täta deadlines, vilket har lett till att en del grupparbete har lagts på helg, och alla har inte kunnat delta. "
%-------



\subsubsection{Förbättringsförslag}
Låta UG ha större ansvar för STLDD som jag skrev tidigare i formuläret.
Något kort om hur man kan skriva ett program som kommunicerar med databasen (hämta sensorvärden etc.). Detta kan ingå i en labb.

%-------
Jag har inga förslag på förbättringar.


\subsubsection{Vetat vad och när du ska göra?}
Som utvecklare var jag inte så delaktig de första veckorna av projektet och då kände jag att jag inte hade koll på vad jag var ansvarig för under hela projektet. Efter att vi kom igång inom UG känns det som om jag har bättre koll på vad jag behöver göra.

PG har varit tydliga med att berätta när de olika delarna ska vara klara. De interna deadlines som vi har satt inom UG har dock inte varit så strikta. Detta kan bero på att vi inte har erfarenhet av android-programmering och därför har svårt att uppskatta hur lång tid det kommer ta.
%-------

Våra veckovisa möten har varit till stor hjälp när det gäller att hålla sig uppdaterad om deadlines och vad som ska göras.
%-------

Det mesta har varit väldokumenterat och trots att jag missat projektmöte har mötesprotokollet gett mig den information. 
%------



\subsubsection{Tidsplan}
Tidsplanen verkar överlag rimlig. Som utvecklare har jag bara varit ansvarig för delar av SRS och SDDD, så jag kan inte säga mycket om de andra dokumenten. En kommentar är som innan att det är svårt att börja koda medan STLDD fortfarande håller på att skrivas. Eftersom man vill följa designen får man ibland ändra i koden när designen uppdateras, vilket har skett några gånger. Det har också skett (mindre) problem med att när STLDD uppdateras kanske den som skriver en klass använder en nyare version än den som använder klassen, vilket skapar problem när de ska sammanfogas.

För SDDD (som inte är klar än) verkar det som om vi har fått lagom med tid.
%------

Det var lite snålt om tid i början när SRS:en och SVVS:en skulle produceras men utöver det så tycker jag planeringen varit väldigt bra.
%-----

En aning täta deadlines men då vi prioriterat att jobba hårdare i början för att kunna vara klara innan tentaveckan så anser jag att det är bra. 
%-------



\subsubsection{Övriga kommentarer}
Ett väl genomfört projekt trots vissa oklarheter från kursens personal har projekter fortsatt i rätt riktning. En bra känsla av att gruppens medlemmar är motiverade och vill göra projektet till något bra har definitivt bidragit mycket till resultatet.


%-------------------------------------------------------------------%
%------------------------------- TG --------------------------------%
%-------------------------------------------------------------------%
\subsection{TG}
\subsubsection{Hur nöjd är du med ditt bidrag, varför?}
"Jag upplever att jag har bidragit ganska mycket till diskussioner och därmed påverkat slutprodukten genom att komma med input och tankar.

Tidsuttaget känns som att det är lite i största laget, då jag ofta upplever att jag stannat kvar efter att alla gått för ""att fixa det sista"", som sen kan bli ganska mycket och ta lång tid.

Sen har vi ju inte kommit till testningsfasen än, så det är ju möjligt att detta ändras, men jag tror inte det... ;)"
%-------

"Jag känner att jag har lagt ganska mycket tid på projektet, både i fråga om faktiskt tid och jämfört med andra. Mina timmar har trots det inte riktigt varit i närheten av de 20 timmar som en halvtidskurs ska ta. Jag känner dock att jag lagt en rimlig mängd timmar för hur mycket tid skolarbete i allmänhet tar.
Jag känner också att jag har haft verktyg för att producera tillräckligt bra dokument, som ju våra uppdrag mest bestått i så här långt. Jag tycker också att jag har arbetat väldigt fokuserat de timmar jag har arbetat."
%--------

Sett till antal timmar jag har bidragit med till projektet är jag väldigt nöjd då jag har prioriterat arbete med gruppen framför många andra projekt i andra kurser jag har läst. Jag har prioriterat att vara med på så många olika steg i projektet som möjligt för att kunna bidra med så mycket som möjligt till gruppen. Sett till kvaliteten på det jag har producerat är jag nöjd. Jag har lagt en hel del tid på självstudier så att jag har kunnat sätta mig in i de olika dokumenten så mycket som möjligt. Detta har gjort att jag har kunnat diskutera projektet på ett relevant sett och kunna ifrågasätta olika beslut i gruppen.
%-------

Jag är på det stora hela nöjd. Jag tycker att mina åsikter och idéer har kommit fram för med mesta. Arbetsbelastningen har varierat från vecka till vecka. Vissa veckor har mycket av arbete gått åt till att felsöka konflikter i GIT vilket jag känt att jag inte kunnat bidra till. Jag är nöjd med mitt bidrag till dokumentskrivande och korrekturläsning. 
%------



\subsubsection{Hur nöjd är du med din grupps prestation, varför?}
"Jag upplever att vi har generellt haft ganska bra kvalité på utprodukterna under projektet, med ganska lite saker som måste fixas under granskningarna.
(Inte för att det inte varit lite kommentarer under granskningarna, utan mycket av det som uppkommit under granskningarna är främst inte saker som vi har kunnat påverka utan mer ""tycke och smak"".)"
%-------

"Hela gruppen har engagerat sig i att ses och försöka producera dokument som vi kan vara nöjda med, och det också för att försöka bli klara i tid till de deadlines som gruppledningen satt upp. Vi har hittills inte heller missat en deadline, och de dokument vi skrivit har klarat de formella granskningarna utan några större ändringar (visserligen tack vare omfattande kommentarer på en av de informella granskningarna).
Generellt tycker jag att ambitionsnivån i gruppen är tillräckligt hög för det arbete vi vill prestera."
%-------

Jag är väldigt nöjd med min grupps arbetsprestation under projektet då vi har producerat väldigt bra dokument som har haft en hög nivå. Arbetet har än så länge fungerat så gott som problemfritt. Vi har ofta arbetat i grupper om 2 eller 3 personer och det har gjort att alla har kunnat bidra med något, oavsett kunskapsnivå angående dokumentskrivning och tester. 
%-------

Alla i gruppen har gjort sitt bästa efter tid och förmåga. Det var någon vecka som någon i gruppen inte alls kunde medverka pga att hen prioriterade andra kurser helt och hållet. För övrigt har samtliga gruppmedlemmar presterat väl. Kvaliteten på det skrivna materialet har skiftat lite på det språkliga. Men det är inget jag anser förminska en individuell prestation. 
%-------



\subsubsection{Arbetsfördelning inom gruppen}
"Jag upplever att kunskapsnivån på gruppmedlemarna är varierande, och då påverkas även vilka uppgifter man kan lösa. T.ex. har vi folk som inte kan Git,och därför inte ens rör det med silvertång... Vilket blir en begränsning.

Indelning av uppgifter sker ofta lite ""ad hoc"" och man får göra lite de uppgifter man vill och kan, vilket kan bli att vissa gör samma saker alltid, för att det är de man kan.

Men överlag så tror jag att alla drar sitt strå till stacken, även om vissa drar större strån. (Hade inte velat byta arbetsuppgift med den som sitter och renskriver granskningssprotokollen exlv.)"
%-------

"Det finns en del kunskapsglapp inom gruppen, där vissa har betydligt mer koll på några av de verktyg vi använder oss av (LaTex samt git), vilket med nödvändighet har gjort att arbetsfördelningen blivit lite ojämn. De som kan mindre har dock lärt sig en hel del under tiden, och ojämlikheten har blivit mindre och mindre.

Det finns också aspekter av att folk har olika scheman, vilket gör att vi när vi har setts har försökt välja tider att ses när så många som möjligt kan. Detta har eventuellt gjort att vissa har fått mindre chanser att arbeta, särskilt om kunskapsluckor gjort det svårt för denne att arbeta på egen hand."
%------

Arbetsfördelningen i gruppen har till stora delar fungerat bra då alla i gruppen har kunnat ses relativt ofta. När inte detta har varit möjligt har de som har kunnat, ställt upp och arbetat med projektet. Det har inte varit några sura miner utan jag, och förhoppningsvis alla andra i min grupp, har känt att vi alla har ställt upp på de tillfällen som vi har kunnat. På grund av den relativt långa projekttiden så känner jag att arbetsfördelningen har jämnat ut sig någorlunda väl på alla medlemmar i gruppen. Det har inte varit några onödiga diskussioner om vem som ska göra vad utan arbetsfördelningen har fungerat problemfritt.
%-------

Jag hävdar att arbetsfördelningen har fungerat "okej". Vi har haft otroligt mycket strul med konflikter i GIT som bara två i gruppen har kunnat hantera. Detta har lett till VÄLDIGT många extra arbetade timmar då övriga gruppmedlemmar antingen har fått  gå hem tidigare eller suttit sysslolösa. När väl att flöt på ordentligt så såg jag inga problem arbetsfördelningen. Var och en skrev på sin sak. Samtliga gruppmedlemmar har medverkat vid avsatt tid för korrekturläsningar. Närvaron på informella och formella granskningar har varit hög. Jag upplever att samtliga gruppmedlemmar har känt för projektet och gjort sitt bästa efter sin förmåga.
%-------



\subsubsection{Hur nöjd är du med produkten?}
"Jag ser två möjligheter att svara på den här frågan, se nedan!

Slutprodukten = appen: Jag upplever att projektet är för enkelt och för litet för att göra på en så här stor kurs, så det känns lite ""mesigt"". Vi har producerat siljarder med papper och dokument och det resulterar i tre vyer och lite funktionalitet.
Därför kan jag ärligt säga att jag inte är nöjd med slutprodukten.

Slutprodukten = slutresultatet för gruppen:
Jag tror att gruppen kommer att leverera en bra slutrapport och slutföra arbetet. Men det kommer nog bli repetetivt och mycket copy-paste skrivande...
Testningen har vi inte kommit igång med, men det känns som att det är lite för enkelt att sitta på en mobil och jämföra layout med en bild (för att vara en avancerad kurs). Hade nog varit mer givande om vi fått skriva testkod och köra lite mer ""hands on"" som man gör på ""riktigt""..."
%-------

"Jag har förtroende för att UG kommer åstadkomma en tillräckligt bra produkt. Utifrån vad jag kan avgöra så verkar det också som om kraven från SG specificerar en produkt jag kan vara nöjd med, och testen som vi har skrivit bör klara av att testa för de kraven på ett tillfredsställande sätt.
Jag tror att det största hotet mot hur nöjd jag blir med produkten är buggar i backend, där vi har begränsad kontroll över vad som händer."
%-------

Jag tror att jag kommer att vara nöjd med slutprodukten när den är färdig. Min grupp har lagt ner mycket tid och arbete på produkten som gör att vår del är noggrant genomarbetad. Förhoppningsvis kommer det inte finnas några fel i slutprodukten men det är svårt att avgöra i dagsläget.  
%--------

"Jag kommer vara nöjd med produkten. Dokumentationen är välskriven och jag tror att appen kommer fungera bra efter specifikationen. 

Jag tycker för övrigt att hela organisationen har fungerat bra. Jag har upplevt en sund stämning under hela projektets gång."
%-------



\subsubsection{Vad har fungerat bra?}
"Jag upplever att kommunikationen från PG och neråt har funkat bra, även om det inte funnits info att sprida :)

Vi har fått ganska bra teamkänsla i TG upplever jag och vi har haft roligt tillsammans, men ändå presterat."
%------

"Jag tycker organisationen från projektgruppens sida har fungerat utmärkt. I testgruppen har vi varit duktiga på att organisera tider då vi kan ses och arbeta tillsammans, och vi har i allmänhet gjort det i tid för att få arbetet klart till den deadline som är satt.
Jag tycker också att gruppledningen har varit utmärkta. Vi har kontinuerligt fått tillgång till viktig information, och jag tycker att veckomötena har fungerat väldigt bra. Det känns bra att alla grupper har kommit till tals under mötena, för att få en känsla för vad som händer på olika ställen. Jag tycker också att gruppledningen har varit duktig på att se till att deras beslut är väl förankrade i gruppen i stort."
%--------

Tidsplaneringen i projektet har fungerar mycket bra vilket jag tycker är en grundläggande del för att ett projekt ska fungera bra. Vi har under hela projektet haft möten som har gett oss en bra uppdatering om hur de går för de andra grupperna. Arbetsplaneringen i min grupp har fungerat väldigt bra då vi aldrig har behövt stressa med att få klart ett dokument. Genom att vi hela tiden har varit färdiga i god tid har vi kunnat korrekturläsa våra dokument och alltid känt oss nöjda med det som vi har lämnat in för granskning. De informella granskningar har än så länge fungerat bra då de flesta har lämnat relevanta synpunkter på förbättringar som bör göras i de olika dokumenten.
%-------

"Möten och planering från PG har varit fantastiskt bra. Helt klart professionell nivå (Inget fjäsk, min personliga åsikt). 

Sammanhållningen har varit bra under hela projektet. Lunchmötena har varit givande. "
%------



\subsubsection{Vad har fungerat mindre bra?}
"Mycket lättare att säga vad som funkat mindre bra!

Kommunikationen från kursledning och neråt har varit katastrofalt dålig. Man hänvisar till varandra i cirklar och vet inte vad som gäller. (Det märks jättetydligt på er i PG att ni inte har information och är frustrerade över detta när vi frågar.)"
%---------

"Det jag kommer att tänka på är att det har funnits brister i problemrapportshanteringen. Vi i testgruppen har behövt ändra i SVVS utifrån ändringar i SRS, men vi har inte alltid fått information om ändringar i SRS och jag tror inte heller att vi fått en chans att kommentera på hur lång tid ändringen i SVVS eller SVVI kommer att ta. Detta har dock blivit bättre sedan tidigare diskussioner och tydligare rutiner.

Jag tycker att från kursledningens sida kunde saker ha gjorts bättre. Det var väldigt kort om tid från det att kursen började till det att man var tvungen att vara igång, och jag tycker inte att det känns som om vi hade verktygen för att komma igång så snabbt. Deras information har inte alltid varit särskilt tydlig, och de har ofta lämnat motstridig information. Ett annat, omfattande problem är att Android studio inte fungerar på skolans datorer. Det är inte okej att studenter tvingas ha tillgång till en dator själva för att kunna fullfölja en kurs."
%-------

Kommunikationen med kursledningen och de personer som har haft olika olika roller i projektet har fungerat mindre bra. 
%-------

"Allt som har fungerat dåligt tillskriver jag kursledningen.
Bristfällig information under de två första veckorna.
Värdelösa dator-laborationen där ingenting fungerade.
Otydliga ""Exercises"".
%------



\subsubsection{Förbättringsförslag}
"Jag kan tycka att det borde ha gjort följande förändringar:
1. Kursledningen borde ha schemallagt vissa tider från början när vi ska jobba med projektet. Då hade vi haft tillgång till en datasal och alla hade kunnat se att då ska vi jobba. Nu har all planering fått ske ad hoc, och vi har fått ta en datasal som varit ledig (vilket bitvis har varit svårt).
Hade räckt med ett kanske två pass i veckan, då man visste att alla var där. Då hade vi kunnat ha möten osv där istället för luncherna.
2. Kursledningen borde ha pratat ihop sig och förbrett sig mer inför kursen! Det har varit extremt tydligt att ingen har haft en helhetsbild. Det är ganska illa att en kurs i att hantera stora projekt inte klarar av att hantera något så enkelt som att installera Android Studio på några datorer...
3. Skit i piazza! Det har bara genererat en massa mail, och vi har inte fått tydliga svar där (ofta på formen goddag yxskaft). Mest bara skapat förvirring och oordning i verksamheten.

Vad vi i gruppen kunde ha gjort annorlunda?
Jag tror vi skulle ha frågat mer i början, och sökt mer experthjälp så hade vi kanske sluppit vissa (i efterhand) ganska uppenbara missar i våra rapporter och programmet."
%-------

"Framför allt tycker jag att kursledningen ska vara bättre förberedda för kursen. Se till att alla handledare vet vad som gäller och är överens om vad som gäller. Se till att utrustningen som behövs fungerar. Se till att man har en rimlig tidsplan. En möjlighet är att ge folk möjlighet att välja grupper innan kursen börjar, eller att helt enkelt sätta preliminära grupper så att studenterna kan komma igång så snabbt som möjligt. Jag tycker också att man ska planera kursen annorlunda, lägg övningarna och labbarna i början. Jag tycker inte att föreläsningarna, möjligen med undantag av föreläsningen om systemet vi utvecklade på, behövde ligga så tidigt. Övningarna behöver också bli tydligare i vad det är man försöker uppnå.

I fråga om vad gruppen har presterat tycker jag att det har varit tillräckligt bra. Eventuellt borde konsensus om hur problemrapporter och git hanteras ha varit större. Det finns också en aspekt av att det har funnits många kommunikationskanaler. Eventuellt borde piazza ha använts mer, men å andra sidan fick vi inte tillgång till det förrän andra kommunikationskanaler redan hade etablerats."
%-------

Jag tycker att föreläsningarna skulle kunna vara bättre. Att all utrustning inte fungerade som den skulle under laborationerna skulle även kunna förbättrats.
%--------

"Skriv tydligare instruktioner för ""exercises"".
Står det på kurshemsidan att det finns förberedelseuppgifter till övningarna så ska det finnas förberedelseuppgifter.
Fixa datorerna i datorsalarna. 
Se över engelskan på instruktioner för laborationer och övningar.
%-------

\subsubsection{Vetat vad och när du ska göra?}
"Mycket möten, och hänger man inte med till fullo på dem, så tror jag lätt att man missat vilka datum som gäller. Nu har jag hängt med på mötena, och därför vetat när allt ska vara klart :)

Framför allt i början så saknade jag en tydligt kommunicerad tidslinje på när allt skulle vara klart.
%------

Första en och en halv-två veckorna tyckte jag det var väldigt otydligt vad som skulle presteras, hur bra det skulle vara och _vad_ det faktiskt var. Detta strukturerades upp till nästa dokument skulle in, och jag känner att vi börjar få lite rutin på det. Jag tror det höga tempot i början och den bristande informationen från kursledningen tillsammans gjorde det svårt att få ordning på saker från början, men i takt med att vår ledningsgrupp har tagit mer och mer kontroll själva så har det blivit bättre och tydligare.
%--------

Jag tycker att projektgruppen har varit väldigt duktiga med att informera om vad som ska göras under projektet. Det har aldrig funnits någon osäkerhet om när saker ska vara klara då det har funnits bra tidsscheman. Preliminära deadlines för olika dokument och informal samt formal reviews tycker jag har hjälp till mycket för att veta hur vårt arbete ligger till i planeringen. Jag uppskattar även att det för varje vecka har stått ungefär hur många timmar som ska läggas ner på projektet så att jag har kunnat planera studieveckan och andra kurser efter det. Inom testgruppen har vi även fört bra diskussioner om vad som ska göras inom de närmsta dagarna vilket har varit väldigt bra.
%-------

De två första veckorna var lite snurriga. Information från kursledningen var bristfällig. 
%---------



\subsubsection{Tidsplan}
"Rimlig och välplanerad, men inte så väl kommunicerad. Det står på flera ställen när saker ska vara klara, men det finns inte samlat och lättillgängligt. (Tog upp detta på informella granskningen.)
Ett tydligt tidsplansdokument på fejjan eller google-drive hade nog underlättat för många. Nu är det dolt i SDP:n.
%-------

"Jag tycker att vi inför alla deadlines har haft tillräckligt mycket tid för att prestera det vi behöver prestera, om bara tillräcklig kunskap funnits från början. Det första dokumentet (SVVS) blev väldigt mycket arbete jämfört med det faktiska dokumentet, men detta beror dels på att vi inte riktigt visste vad vi skulle göra, dels inte hade tid att vänta tills dokumentet vi skulle basera vårt dokument på (SRS) blev klart. Detta gjorde att det blev ganska många iterationer av skrivande som ändrades flera gånger utifrån SRS:en, antagligen mer än nödvändigt.

Jag känner också att det funnits en brist i att många av de övningar och labbar (alla) som funnits för att öva sig i färdigheter man behöver för de olika projektdelarna har legat den sista tredjedelen eller liknande av perioden där man behöver kunna dem. Alltså, man får knappt en chans till handledning eller en chans att öva sig i färdigheter förrän man måste kunna dem. Detta var särskilt tydligt i fråga om labbarna."
%-------

Tidsplanen var tuff i början men jag uppskattade det då det har känts som om vi har varit i fas under hela projektet vilket har känts väldigt bra. Efter den första veckan som var lite stressig så tycker jag att tidsplanen har varit bra då det har varit någorlunda mycket arbete med projektet varje vecka. Jag tycker också att tidsplanen har varit bra planerad då den har haft utrymme för plötsliga problem vilket har känts bra. Jag uppskattar att vi planerar att vara klara med projektet innan läsvecka 8. 
%------

Tidsplanen har varit rimlig. Många täta deadlines har gjort att projektet aldrig stannat upp. Alltid skönt när saker och ting går framåt. 
%--------



\subsubsection{Övriga kommentarer}






\end{document}