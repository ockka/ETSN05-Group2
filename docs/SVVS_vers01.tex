\documentclass[a4paper]{article}

\usepackage[english]{babel}
\usepackage[utf8]{inputenc}
\usepackage{amsmath}
\usepackage{graphicx}
\usepackage{hyperref}
\usepackage[colorinlistoftodos]{todonotes}
\usepackage{color} % Used to write in red,blue,green etc.
\usepackage{float}
\usepackage[titletoc]{appendix}

%-------------------------TITLE-----------------------------------------

\title{SVVS - Software Verification and Validation Specification}

\date{\today}

\begin{document}

\maketitle


\pagebreak
\setcounter{tocdepth}{2}
\tableofcontents
\pagebreak

%--------------------Introduction ---------------------------------------
\section{Introduction}
The Software Verification and Validation Specification (SVVS) describes all tests and reviews conducted during the development of the project \textbf{(project name here)}. \textcolor{red}{Write something more here, tell  the reader about the various appendices etc...}

%--------------------Reference Documents ---------------------------------------
\section{Reference documents}
Reference document: PUS15001.

%--------------------Testing ---------------------------------------
\section{Testing}
In order to validate and verify that the system fulfills the customers expectations, various tests are conducted during the development phases. Types of test i.e. unit tests, function tests etc. can be divided into two categories: white-box tests and black-box tests, which are described below.   

\subsection{White-box testing}
In a white-box test the developer tests the internal structure of the system, by means of making sure that all lines/units of code are executed at least once.

\subsection{Black-box testing}
A function test or black-box test examines the functionality of a system from an external point of view. Combinations of input data are chosen and the output data from the system is compared to a specification.


%--------------------Reviews ---------------------------------------
\section{Reviews}

\subsection{Informal Reviews}
Informal Reviews are to be held at least ten hours before the deadline for the Formal Review. The documents up for review should be made available in the project library at least 24 hours before the informal review.

The Test group is responsible for appointing a coordinator before each informal review, whose responsibility it is to select reviewers for each document. The coordinator should make these selections such that the author of a document will not review the same document. Among these reviewers, one should be randomly appointed review group leader. The coordinator should also appoint one moderator and one secretary for the review. These selections should be emailed to the entire project group at least 24 hours before the informal review.

Each reviewer is responsible for reading their appointed document and sending an email with their comments and issues to their review group leader at least two hours before the review.

During the review, the moderator is responsible for managing time to make sure that each document can be reviewed during the allotted time frame. The secretary is responsible for correctly recording the problems that surface during the review. Each review group leader is responsible for presenting the issues submitted to them from the rest of the group.

\begin{enumerate}


\item Software Specification Review (SSR) - \textcolor{red}{DATE HERE}
\begin{itemize}
\item SDP
\item SRS
\item SVVS
\end{itemize}


\item Preliminary Design Review (PDR) - \textcolor{red}{DATE HERE}
\begin{itemize}
\item SVVI
\item STLDD
\end{itemize}


\item Product Review (PR) - \textcolor{red}{DATE HERE}
\begin{itemize}
\item SVVR
\item SSD
\item PFR
\end{itemize}


\end{enumerate}

%--------------------Test environments ---------------------------------------
\section{Test environments}
\textit{I SVVS skall även anges vad som skall testas i utvecklingsmiljön respektive på målmaskinen.}

%--------------------Test cases ----------------------------------------------
\section{Test cases}

\subsection{Unit tests}
Done by the developer group, white-box and black-box tests.

\subsection{Function tests}

\subsection{System tests}
Complete test of the system...

\subsection{Regression tests}
Used to check consistency with old code.  

\subsection{Acceptance tests}
Test done by the customer...

%--------------------Appendix A: Function test Specification ----------------
\newpage
\begin{appendices}

\section{Function test specification}
This is appendix A where we list all function test specifications.

\subsection{This is a subsection}

\subsubsection{This is a subsubsection}

\end{appendices}

%--------------------Appendix B: System test Specification ----------------
\newpage
\begin{appendices}

\section{System test specification}
This is appendix B where we list all system test specifications.

\subsection{This is a subsection}

\subsubsection{This is a subsubsection}

\end{appendices}

%--------------------Appendix C: Regression test Specification ----------------
\newpage
\begin{appendices}
\section{Regression test specification}
This is appendix C where we list all regression test specifications.

\subsection{This is a subsection}

\subsubsection{This is a subsubsection}
\end{appendices}


\end{document}