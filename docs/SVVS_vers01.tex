\documentclass[a4paper]{article}

\usepackage[english]{babel}
\usepackage[utf8]{inputenc}
\usepackage{amsmath}
\usepackage{enumitem}%can be used for automatic numbering of requirements/tests.
\usepackage{graphicx}
\usepackage{hyperref}
\usepackage[colorinlistoftodos]{todonotes}
\usepackage{color} % Used to write in red,blue,green etc.
\usepackage{float}
\usepackage[titletoc]{appendix}

%-------------------------TITLE-----------------------------------------

\title{SVVS - Software Verification and Validation Specification}

\date{\today}
\author{Test Group 2}
\begin{document}

\maketitle


\pagebreak
\setcounter{tocdepth}{2}
\tableofcontents
\pagebreak


%--------------------Introduction ---------------------------------------
\section{Introduction}
The Software Verification and Validation Specification (SVVS) describes all tests and reviews conducted during the development of the project \textbf{(project name here)}. The document acts as a base for the SVVI (Software Verification and Validation Instruction).
In the main document, there are instructions for reviews as well as information about what types of tests should be performed and by whom. In the appendices, test cases are specified.

%--------------------Reference Documents ---------------------------------------
\section{Reference documents}
\begin{enumerate}
\item SRS \label{refdocs:srs}
\item Programvaruutveckling för stora system - Projekthandledning (\textit{Institutionen för datavetenskap}, Lunds Univeritet 2015) \label{refdocs:projekthandledning}
\end{enumerate}

%--------------------Definitions-------------------------------------------
\section{Definitions}
\begin{description}
\item[SDP] Software Development Plan
\item[SRS] Software Requirements Specification
\item[SVVS] Software Verification and Validation Specification
\item[SVVI] Software Verification and Validation Instruction
\item[STLDD] Software Top Level Design Document
\item[SDDD] Software Detailed Design Document
\item[SVVR] Software Verification and Validation Report
\item[SSD] Software Specification Document
\item[PFR] Project Final Report
\item[PG] The roject Group Leaders
\item[MVD] Minimum Viable Device, part of the back end system
\end{description}
%--------------------Testing ---------------------------------------
\section{Testing}
In order to validate and verify that the system fulfills the customers expectations, various tests are conducted during the development phases. Types of test i.e. unit tests, function tests etc. can be divided into two categories: white-box tests and black-box tests, which are described below.   

\subsection{White-box testing}
In a white-box test the developer tests the internal structure of the system, by means of making sure that all lines/units of code are executed at least once.

\subsection{Black-box testing}
A function test or black-box test examines the functionality of a system from an external point of view. Combinations of input data are chosen and the output data from the system is compared to a specification.


%--------------------Reviews ---------------------------------------
\section{Reviews}

\subsection{Formal Reviews}
Formal reviews are conducted according to ref \ref{refdocs:projekthandledning}.

The scheduled formal reviews in this project:

\begin{enumerate}

\item Software Specification Review (SSR)
\begin{itemize}
\item SDP
\item SRS
\item SVVS
\end{itemize}


\item Preliminary Design Review (PDR)
\begin{itemize}
\item SVVI
\item STLDD
\end{itemize}


\item Product Review (PR)
\begin{itemize}
\item SVVR
\item SSD
\item PFR
\end{itemize}


\end{enumerate}


\subsection{Informal Reviews}
Informal Reviews are to be held at least ten hours before the deadline for the Formal Review. The documents up for review should be made available in the project library at least 24 hours before the informal review.

PG are the responsible coordinators for the informal review.

%--------------------Test environments ---------------------------------------
\section{Test environments}
All tests specified in this document should be performed on appropriate emulator software. If possible, the tests should also be run on a physical device.

%--------------------Test cases ----------------------------------------------
\section{Types of tests}

\subsection{Unit tests}
The developer group should perform unit tests, both black-box tests and white-box tests. These tests should be performed continuously during development.

\subsection{Function tests}
All functions of the system should be tested according to appendix \ref{appendix:section:functiontest}. The test group is responsible for  performing these tests.

\subsection{System tests}
The entire system should be tested according to appendix \ref{appendix:section:systemtest}. The test group is responsible for performing these tests. 

\subsection{Regression tests}
After any changes, all System tests should be performed again to ensure that previously functional code is still working. The test group is responsible for performing these tests.

\subsection{Acceptance tests}
Acceptance tests are performed by the costumer before completion of the project. The costumer is responsible for selecting tests to ensure that the software meets their demands.

%--------------------Appendix A: Function test Specification ----------------
\newpage
\begin{appendices}

\section{Function test specification} \label{appendix:section:functiontest}
This is appendix A where we list all function test specifications.

\subsection{General test}
%
% Detta avsnittet bör granskas så att alla tester är formulerade i presens. Dvs. "Appen gör så här", "Det funkar på detta viset", dvs inte "Den ska göra så här". = Det ska vara ett påstående som går att svar med ja eller nej.
%
\begin{enumerate}[label=\bfseries Test \thesubsection.\arabic*,labelwidth=\widthof{\text{Test 10.10.10}},leftmargin=!]
	\item The first screen that is shown on startup is the MyDevicesView. (Req 5.2.1)
	%\item After pressing "Get Devices" button a list of temperature sensors are shown. (Req 5.1.1)
	%\item After pressing  "Get Devices" button a list of lightbulbs are shown on a list. (Req 5.1.1)
    %\item With no connected devices, then a pop up message with the text "No devices found" should be shown. (Req 5.1.1)
	%\item The lightbulb should turn off, when the onoff button is pressed from the controll device screen. (Req 5.1.2)
    %\item With no lightbulb connected, a error message should appear with the text "No device selected". (Req 5.1.2) \label{appfunctest:no_lightbulb}
    %\item The user can change the color of the light from red to green in the controll device screen. (Req 5.1.3)
    %\item If no device is selected the error message should appear as in test \ref{appfunctest:no_lightbulb}. (Req 5.1.3)
    %\item The user can extract temperature data from the sensor using the Get button on the Sensor View. (Req 5.1.4)
    %\item If no device is selected then the the error message should appear as in test \ref{appfunctest:no_lightbulb}. (Req 5.1.4)
    %\item If no data is avalaible, the a pop upp message should appear and read: "No sensor data available". (Req 5.1.4)
\end{enumerate}

\subsection{MyDevices View Tests}
\begin{enumerate}[label=\bfseries Test \thesubsection.\arabic*,labelwidth=\widthof{\text{Test 10.10.10}},leftmargin=!]
	\item The list on the MyDevices view is scrollable. (Req 5.2.1)
    \item The items on he list is be selectable. (Req 5.2.3)
    \item When no device is selected, then the "Controll device" button is not clickable. (Req 5.2.4)
    \item The screen displays the number of items in the list. (Req 5.2.5)
    \item When the application is started the list of available devices is be empty. (Req 5.2.6)
    \item The devices are named in accordance with relevant documen. (Which document is unknown at present...) (Req 5.2.7)
    \item The "Get Devices" button initiates a scan with the MVD. (Req 5.2.8)
    \item The scan with the MVD is halted after X seconds. (Exact number of seconds is unspecified in SRS at the moment.) (Req 5.2.9)
\end{enumerate}

\subsection{Sensor View Tests}
\begin{enumerate}[label=\bfseries Test \thesubsection.\arabic*,labelwidth=\widthof{\text{Test 10.10.10}},leftmargin=!]
\item Blarg!
\end{enumerate}

\subsection{Lightbulb View}
\begin{enumerate}[label=\bfseries Test \thesubsection.\arabic*,labelwidth=\widthof{\text{Test 10.10.10}},leftmargin=!]
	\item The Light Bulb View opens when a light bulb is chosen in the MyDevices View. (req 5.4.1)
    \item The name (Which document is unknown at present...), is shown in the top of the View.
    \item
\end{enumerate}	

\subsubsection{This is a subsubsection}

%--------------------Appendix B: System test Specification ----------------
\newpage

\section{System test specification} \label{appendix:section:systemtest}
This is appendix B where we list all system test specifications.

\subsection{Use cases as defined in the SRS}
\begin{enumerate}[label=\bfseries Test \thesubsection.\arabic*,labelwidth=\widthof{\text{Test 10.10.10}},leftmargin=!]
	\item Scenario 5.1.1 from \ref{refdocs:srs} is supported.
	\item The exception specified in scenario 5.1.1 can be generated by removing the lightbulb and the sensor from MVD range.
	
	\item Scenario 5.1.2 is supported.
	\item The exception specified in scenario 5.1.2 can be generated by not selecting any device.
	
	\item Scenario 5.1.3 is supported.
	\item An exception can be generated by not selecting any device.
	\item An exception can be generated by not filling out all fields (or it will be set to zero as by requirement 5.4.11)
	
	\item Scenario 5.1.2 is supported.
	\item An exception can be generated by not selecting any device.
	
	\item Scenario 5.1.4 is supported.
	\item An exception can be generated by not selecting any device.
	 
\end{enumerate}

\subsection{This is a subsection}

\subsubsection{This is a subsubsection}

%--------------------Appendix C: Regression test Specification ----------------
\newpage

\section{Regression test specification}
This is appendix C where we list all regression test specifications.

\subsection{This is a subsection}

\subsubsection{This is a subsubsection}
\end{appendices}


\end{document}