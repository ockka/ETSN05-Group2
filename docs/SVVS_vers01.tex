\documentclass[a4paper]{article}
\usepackage[utf8]{inputenc}
\usepackage{amsmath}
\usepackage{amsfonts}
\usepackage{amssymb}
\usepackage{graphicx}
\usepackage{geometry}
\usepackage[english]{babel}
\usepackage{enumitem}%can be used for automatic numbering of requirements/tests.
\usepackage{hyperref}
\usepackage[colorinlistoftodos]{todonotes}
\usepackage{color} % Used to write in red,blue,green etc.
\usepackage{float}
\usepackage[titletoc]{appendix}

\newlength{\testlabellength}
\settowidth{\testlabellength}{Test 100.10.10}
\newenvironment{testlist}{\begin{enumerate}[label=\bfseries Test \thesubsection.\arabic* , labelindent=0pt, labelwidth=\testlabellength , leftmargin=2cm]}{\end{enumerate}}

\title{SVVS - Software Verification and Validation Specification}
\author{Testgroup, Team 2}


%-------------------------TITLE-----------------------------------------

\begin{document}
\begin{titlepage}
\newgeometry{left=2cm,top=1cm,right=2cm}
\newcommand{\HRule}{\rule{\linewidth}{0.5mm}}

\begin{minipage}{0.5\textwidth}
\begin{flushleft} % Responsible persons, write on separate lines
\textit{Responsible for this document:}\\
Oskar Fällström %Not entirely sure who this should be, possibly project leaders as well?
\end{flushleft}
\end{minipage}
~
\begin{minipage}{0.4\textwidth}
\begin{flushright}
PUSS154213 0.1 %Dokumentnummer enl. projekthandledning s. 22-23 och insidan av pärmen
\today
\end{flushright}
\end{minipage}\\[3cm]

\centering
\textsc{\LARGE Team 2}\\[0.5cm]

\HRule \\[0.4cm]
{ \huge \bfseries Software Verification and Validation Specification}\\[0.4cm] % Title of your document
\HRule \\[1.5cm]

\vfill
\begin{flushleft}
\textit{Authors of this document:}\\
Måns Andersson \\
Hanna Autio \\
Moa Eklöf \\
Oskar Fällström \\
Ulf Hörndahl \\
Jonathan Lundholm
\end{flushleft}


\end{titlepage}
\pagenumbering{gobble}
\setcounter{tocdepth}{2}
\tableofcontents
\newpage
\pagenumbering{arabic}

%--------------------Introduction ---------------------------------------
\section{Introduction}
The Software Verification and Validation Specification (SVVS) describes all tests and reviews conducted during the development of the project \textbf{(project name here)}. The document acts as a base for the SVVI (Software Verification and Validation Instruction).
In the main document, there are instructions for reviews as well as information about what types of tests should be performed and by whom. In the appendices, test cases are specified.

%--------------------Reference Documents ---------------------------------------
\section{Reference documents}
\begin{enumerate}
\item PUSS1542xx - System Requirements Specification for the current project \label{refdocs:srs}
\item Programvaruutveckling för stora system - Projekthandledning (\textit{Institutionen för datavetenskap}, Lunds Univeritet 2015) \label{refdocs:projekthandledning}
\end{enumerate}

%--------------------Definitions-------------------------------------------
\section{Definitions}
\begin{description}
\item[SDP] Software Development Plan
\item[SRS] Software Requirements Specification
\item[SVVS] Software Verification and Validation Specification
\item[SVVI] Software Verification and Validation Instruction
\item[STLDD] Software Top Level Design Document
\item[SDDD] Software Detailed Design Document
\item[SVVR] Software Verification and Validation Report
\item[SSD] Software Specification Document
\item[PFR] Project Final Report
\item[PG] Project Group Leaders
\end{description}
%--------------------Testing ---------------------------------------
\section{Testing}
In order to validate and verify that the system fulfills the customers expectations, various tests are conducted during the development phases. Types of test i.e. unit tests, function tests etc. can be divided into two categories: white-box tests and black-box tests, which are described below.   

\subsection{White-box testing}
In a white-box test the developer tests the internal structure of the system, by means of making sure that all lines/units of code are executed at least once.

\subsection{Black-box testing}
A function test or black-box test examines the functionality of a system from an external point of view. Combinations of input data are chosen and the output data from the system is compared to a specification.


%--------------------Reviews ---------------------------------------
\section{Reviews}

\subsection{Formal Reviews}
Formal reviews are conducted according to ref \ref{refdocs:projekthandledning}.

The scheduled formal reviews, and the documents up for review, in this project are:

\begin{enumerate}

\item Software Specification Review (SSR)
\begin{itemize}
\item SDP
\item SRS
\item SVVS
\end{itemize}


\item Preliminary Design Review (PDR)
\begin{itemize}
\item SVVI
\item STLDD
\end{itemize}


\item Product Review (PR)
\begin{itemize}
\item SVVR
\item SSD
\item PFR
\end{itemize}


\end{enumerate}


\subsection{Informal Reviews}
Informal Reviews are to be held at least ten hours before the deadline for the Formal Review. The documents up for review should be made available in the project library at least 24 hours before the informal review.

PG are the responsible coordinators for the informal review.

%--------------------Test environments ---------------------------------------
\section{Test environments}
All tests specified in this document should be performed on appropriate emulator software. If possible, the tests should also be run on a physical device.

%--------------------Test cases ----------------------------------------------
\section{Types of tests}

\subsection{Unit tests}
The developer group should perform unit tests, both black-box tests and white-box tests. These tests should be performed continuously during development.

\subsection{Function tests}
All functions of the system should be tested according to appendix \ref{appendix:section:functiontest}. The test group is responsible for  performing these tests.

\subsection{System tests}
The entire system should be tested according to appendix \ref{appendix:section:systemtest}. The test group is responsible for performing these tests. 

\subsection{Regression tests}
After any changes, all System tests should be performed again to ensure that previously functional code is still working. The test group is responsible for performing these tests.

\subsection{Acceptance tests}
Acceptance tests are performed by the costumer before completion of the project. The costumer is responsible for selecting tests to ensure that the software meets their demands.

%--------------------Appendix A: Function test Specification ----------------
\newpage
\begin{appendices}

\section{Function test specification} \label{appendix:section:functiontest}
This is appendix A where we list all function test specifications.

%
% Detta avsnittet bör granskas så att alla tester är formulerade i presens. Dvs. "Appen gör så här", "Det funkar på detta viset", dvs inte "Den ska göra så här". = Det ska vara ett påstående som går att svar med ja eller nej.
%

\subsection{MyDevices View Tests}
\begin{testlist}
	\item The list on the MyDevices view is scrollable. (Req 5.2.1)
    \item The items on he list is be selectable. (Req 5.2.3)
    \item When no device is selected, then the "Controll device" button is not clickable. (Req 5.2.4)
    \item The screen displays the number of items in the list. (Req 5.2.5)
    \item When the application is started the list of available devices is be empty. (Req 5.2.6)
    \item The devices are named in accordance with relevant documen. (Which document is unknown at present...) (Req 5.2.7)
    \item The "Get Devices" button initiates a scan with the MVD. (Req 5.2.8)
    \item The scan with the MVD is halted after X seconds. (Exact number of seconds is unspecified in SRS at the moment.) (Req 5.2.9)
\end{testlist}

\subsection{Sensor View Tests}
\begin{testlist}
\item Blarg!
\end{testlist}

\subsection{Lightbulb View}
\begin{testlist}
	\item The Light Bulb View opens when a light bulb is chosen in the MyDevices View. (req 5.4.1)
    \item The name (Which document is unknown at present...), is shown in the top of the View.
    \item
\end{testlist}	

\subsubsection{This is a subsubsection}

%--------------------Appendix B: System test Specification ----------------
\newpage

\section{System test specification} \label{appendix:section:systemtest}
This is appendix B where we list all system test specifications.

\subsection{Use Cases}
\begin{testlist}
	\item Use case 5.1.1 is supported
	\item The exception described in use case 5.1.1 can be generated by removing the devices from range while attempting use case 5.1.1
	
	\item Use case 5.1.2 is supported
	\item The exception described in use case 5.1.2 can be generated by not selecting any device while attempting use case 5.1.2
	
	\item Use case 5.1.3 is supported
	\item Exception 5.1.3.2 can be generated by not selecting any device while attempting use case 5.1.3
	\item Exception 5.1.3.3 can be generated by not filling out at least one field while attempting use case	5.1.3 (alternatively, this will follow requirement 5.4.11)
	
	\item Use case 5.1.2 is supported.
	\item The exception described ni use case 5.1.2 can be generated by not selecting any device.
	
	\item Use case 5.1.4 is supported
	\item Exception 5.1.4.2 can be generated by not selecting any device while attempting use case 5.1.4
	
\end{testlist}

\subsection{Appearance and Layout}

\begin{testlist}
	\item The first screen that is shown on startup is the MyDevicesView. (Req 5.2.1)
	\item The layout of the MyDevices View resembles figure x in reference \ref{refdocs:srs}.
	\item The layout of the Sensor View resembles figure xx in reference \ref{refdocs:srs}.
	\item The layout of the Light Bulb View resembles figure xxx in reference \ref{refdocs:srs}.
\end{testlist}

%--------------------Appendix C: Regression test Specification ----------------
\newpage
\section{Regression test specification}
This is appendix C where we list all regression test specifications.

\subsection{This is a subsection}

\subsubsection{This is a subsubsection}
\end{appendices}


\end{document}