\documentclass[a4paper]{article}
\usepackage[utf8]{inputenc}
\usepackage[english]{babel}
\usepackage{amsmath}
\usepackage{amsfonts}
\usepackage{amssymb}
\usepackage{graphicx}
\usepackage{geometry}
\usepackage{footnote}

\makesavenoteenv{tabular}


\title{SSD - Software Specification Document}
\author{Team 2}

\begin{document}
\begin{titlepage}
\newgeometry{left=2cm,top=1cm,right=2cm}
\newcommand{\HRule}{\rule{\linewidth}{0.5mm}}

\begin{minipage}{0.5\textwidth}
\begin{flushleft} % Responsible persons, write on separate lines
\textit{Responsible for this document:}\\
Emma Albertz \\
Linnéa Claesson
\end{flushleft}
\end{minipage}
~
\begin{minipage}{0.4\textwidth}
\begin{flushright}
PUSS154218 v1.0
\today
\end{flushright}
\end{minipage}\\[3cm]

\centering
\textsc{\LARGE Team 2}\\[0.5cm]

\HRule \\[0.4cm]
{ \huge \bfseries Software Specification Document}\\[0.4cm] % Title of your document
\HRule \\[1.5cm]

\vfill
\begin{flushleft}
%Authors, write on separate lines
\textit{Authors of this document:}\\
Emma Albertz \\
Linnéa Claesson
\end{flushleft}



\end{titlepage}
\pagenumbering{gobble}



\begin{center}
\textit{\large Version History}

    \begin{tabular}{ | l | l | l | p{5cm} |}
    \hline
    \textbf{Version} & \textbf{Date} & \textbf{Responsible} & \textbf{Description} \\ \hline
    1.0 & 2015-10-14 & EA, LC & Baseline\\ \hline
    \end{tabular}
\end{center}



\setcounter{tocdepth}{2}
\tableofcontents
\newpage
\pagenumbering{arabic}


%---------------------- Introduction ---------------------------------------------%
\section{Introduction}

The intent of this document is to describe the different parts of the delivered system, with current version number and appendices of each document. Additionally, limitations of the delivered systems and larger changes made after baseline of the System Requirements Specification are included.



%---------------------- Delivered System -----------------------------------------%
\section{Delivered System}

The delivered system consists of the following documents/parts:

\begin{center}
\begin{tabular}{| l | c | c | p{5cm} |}
    \hline
    \textbf{Doc./part} & \textbf{Doc. number} & \textbf{Version} & \textbf{Comments} \\ \hline
    SDP & 11 & 1.0 & With appendices: Veckoschema PUSS 154251 and Gantt-schema PUSS154252 \\ \hline
    SRS & 12 & 1.4 & \\ \hline
    SVVS & 13 & 1.4 & With appendix: Test Matrices PUSS154253 \\ \hline
    STLDD & 14 & 1.1 & \\ \hline
    SVVI & 15 & 1.6 & \\ \hline
    SDDD & 16 & 1.0 & appGrupp2.apk\\ \hline
    SVVR & 17 & 1.0 & \\ \hline
    SSD & 18 & 1.0 & \\ \hline
    PFR & 19 & 1.0 & \\ \hline

\end{tabular}
\end{center}

%---------------------- Limitations ----------------------------------------------%
\section{Limitations}


\subsection{System Objective}
The intention was that the delivered system would be launched as an application on Google Play. Due to bugs in the back end of the system, this is not reasonable before they are fixed. 

The system has not yet been tested on real users, i.e. users not connected with this project, since the system has been assessed not fit for the market as long as the bugs in the back end remain.  

\subsection{Differences Compared to the Software Requirements Specification (SRS)}
Minor changes regarding error handling and number of detectable devices have been made to the SRS after baseline was set. This however, does not affect the functionality or customer experience of the final product. See status report of SRS for more details. 


%------------------- Installation Instructions -----------------------------------%
\section{Installation Instructions}
Since the application is not released on Google Play, the application needs to be installed using the application file \textbf{appGrupp2.apk}. 

\begin{enumerate}
\item Make sure you have an Android phone running at least version 4.1.
\item Connect the phone to a computer via USB and make sure the computer can find it.
\item Transfer \textbf{appGrupp2.apk} to the phone.
\item Browse the phone for the \textbf{appGrupp2.apk} file with a file explorer app and press on it to install.
\item The application should now be installed and ready to use.
\end{enumerate}

\end{document}