\documentclass[a4paper]{article}

\usepackage[english]{babel}
\usepackage[utf8]{inputenc}
\usepackage{amsmath}
\usepackage{verbatim}
\usepackage{graphicx}
\usepackage{hyperref}
\usepackage[colorinlistoftodos]{todonotes}
\usepackage{float}
\usepackage{geometry}

\title{STLDD - Software Top Level Design Document}
\author{Team 2}

\begin{document}
	\begin{titlepage}
		\newgeometry{left=2cm,top=1cm,right=2cm}
		\newcommand{\HRule}{\rule{\linewidth}{0.5mm}}
		
		\begin{minipage}{0.5\textwidth}
			\begin{flushleft} % Responsible persons, write on separate lines
				\textit{Responsible for this document:}\\
				Oscar Axelsson \\
				Daniel Olsson \\
				Jacob Mejvik
			\end{flushleft}
		\end{minipage}
		~
		\begin{minipage}{0.4\textwidth}
			\begin{flushright}
				PUSS154214
				\today
			\end{flushright}
		\end{minipage}\\[3cm]
		
		\centering
		\textsc{\LARGE Team 2}\\[0.5cm]
		
		\HRule \\[0.4cm]
		{ \huge \bfseries Software Top Level Design Document}\\[0.4cm] % Title of your document
		\HRule \\[1.5cm]
		
		\vfill
		\begin{flushleft}
			%Authors, write on separate lines
			\textit{Authors of this document:}\\
			Jacob Mejvik \\
			Oscar Axelsson \\
			Daniel Olsson
		\end{flushleft}
		
	\end{titlepage}
	\pagenumbering{gobble}
	\setcounter{tocdepth}{2}
	
	\begin{center}
		\textit{\large Version History}
		
		\begin{tabular}{ | l | l | l | p{5cm} |}
			\hline
			\textbf{Version} 	& \textbf{Date} 	& \textbf{Responsible} 	& \textbf{Description} 		\\ \hline
			0.9				 	& 230915 			& DO, JM, OA			&  Working on. 				\\ \hline
		\end{tabular}
	\end{center}
	
	
	\tableofcontents
	\newpage
	\pagenumbering{arabic}
	
	\section{Introduction}
	This document describes the top level design of the Lamp Controller Application. The application can connect to a light bulb and a sensor device, these devices can also be controlled and data can be received using a MVD. The application is developed as a project within the course "Software Development for Large Systems - ETSN05" at LTH.
	
	\section{Reference Documents}
	SRS - Software Requirements Specification, PUSS154212 v1.1
	
	
	\section{Overview}
	The main purpose of the Lamp Controller Application is to provide an graphical interface used to control a light bulb and a sensor device from an MVD via a REST API provided by the backend. The application will provide three different views to detect and control the different devices. A UML diagram have been created to visualize the design of a system figure \ref{fig:uml}. Several sequence diagram have been created to visualize the interaction and can be viewed in figures 2-5.
	
	\subsection{Controller Application}
	\begin{itemize}
		\item{\textbf{BaseActivity:}} 
		The BaseActivity for the application that will connect the Activities with the NetworkManager.
		\item{\textbf{MyDeviceActivity:}} 
		Is the start screen of the application. Will contain a ListView and has methods for detecting devices and control a device that was selected from the ListView.
		\item{\textbf{DeviceListAdapter:}} 
		Adapter that handles the list in MyDevicesActivity, the ListView can display any data provided that it is wrapped in a ListAdapter. 
		\item{\textbf{Device:}} 
		Is an Abstract class that controlls the two devices we are handling. 
		\item{\textbf{SensorDevice:}} 
		Class containing the SensorDevice information.
		\item{\textbf{LightBulb:}}
		Class containing the LightBulb information.
		\item{\textbf{DeviceActivity:}} 
		Is an abstract class that holds the shared parameters deviceName and macAddress. It also contains an abstract method toggle that controls the on/off switch on the devices.
		\item{\textbf{SensorDeviceActivity:}} 
		Is the controller in the interaction with the user in the SensorDevice View. Controls the TextView fields and the buttons that will retrieve information regarding the sensors from NetworkManager.
		\item{\textbf{LightBulbActivity:}} 
		Is the controller in the interaction with the user in the LightBulb View. Controls the EditText fields and the buttons that will retrieve and send information from/to the NetworkManager.
		
		\item{\textbf{NetworkManager:}} 
		Handles all the communication with the API. The different methods for controlling both receiving and setting data.
		\item{\textbf{SensorValues:}} 
		A class for the information of the different sensors.
		
	\end{itemize}

	\subsection{Packages}
	\begin{itemize}
		\item Network \\
        Handles the network communication and contains the class NetworkManager.
		\item Activity
        Handles the Activities for the different views. Contains the classes BaseActivity, DeviceActivity, MyDeviceActivity, SensorDeviceActivity and LighBulbActivity.
		\item Adapter
        The Adapter to the ListView used in MyDeviceActivity contains the class DeviceListAdapter.
		\item Model
        Handels the information about the different devices contains the classes Device, SensorDevice and LightBulb.
		\item Sensor
        Handels and controls the Sensor values contains the classes SensorValues, Temperature, Pressure, Humidity, Magnetic, Gyroscopic and Accelerometer.
	\end{itemize}
\subsection{UML diagram}
	\begin{figure}[H]
    \centering
    \includegraphics[width=0.8\textwidth]{class_diagram.png}
    \caption{The design of the system.}
    \label{fig:uml}
\end{figure}

	
	\subsection{Sequence diagrams}
	
	\begin{figure}[H]
    \centering
    \includegraphics[width=0.6\textwidth]{seq.png}
    \caption{Opening application and detecting devices.}
    \label{fig:seq}
\end{figure}

\begin{figure}[H]
    \centering
    \includegraphics[width=0.6\textwidth]{seq1.png}
    \caption{Turn on the light bulb with the user located in the LightBulb view.}
    \label{fig:seq1}
\end{figure}

\begin{figure}[H]
    \centering
    \includegraphics[width=0.6\textwidth]{seq2.png}
    \caption{Get all sensor data with the user located in the SensorDevice view.}
    \label{fig:seq2}
\end{figure}

\begin{figure}[H]
    \centering
    \includegraphics[width=0.6\textwidth]{seq3.png}
    \caption{Set color of light bulb with the user located in the LightBulb view and the backend is unresponsive.}
    \label{fig:seq3}
\end{figure}
	
\end{document}